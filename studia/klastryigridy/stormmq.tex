%%%%%%%%%%%%%%%%%%%%%%%%%%%%%%%%%%%%%%%%%
% Short Sectioned Assignment
% LaTeX Template
% Version 1.0 (5/5/12)
%
% This template has been downloaded from:
% http://www.LaTeXTemplates.com
%
% Original author:
% Frits Wenneker (http://www.howtotex.com)
%
% License:
% CC BY-NC-SA 3.0 (http://creativecommons.org/licenses/by-nc-sa/3.0/)
%
%%%%%%%%%%%%%%%%%%%%%%%%%%%%%%%%%%%%%%%%%

%----------------------------------------------------------------------------------------
%   PACKAGES AND OTHER DOCUMENT CONFIGURATIONS
%----------------------------------------------------------------------------------------

\documentclass[paper=a4, fontsize=11pt]{scrartcl} % A4 paper and 11pt font size

\usepackage[T1]{fontenc} % Use 10-bit encoding that has 256 glyphs
\usepackage{polski}
\usepackage[utf8]{inputenc}
\usepackage[polish]{babel}
\usepackage{amsmath,amsfonts,amsthm} % Math packages
\usepackage{pgfplots} % Math packages
\usepackage{lipsum} % Used for inserting dummy 'Lorem ipsum' text into the template
\usepackage{enumerate}
\usepackage{sectsty} % Allows customizing section commands
\usepackage{fancyhdr} % Custom headers and footers
\pagestyle{fancyplain} % Makes all pages in the document conform to the custom headers and footers
\fancyhead{} % No page header - if you want one, create it in the same way as the footers below
\fancyfoot[L]{} % Empty left footer
\fancyfoot[C]{} % Empty center footer
\fancyfoot[R]{\thepage} % Page numbering for right footer
\renewcommand{\headrulewidth}{0pt} % Remove header underlines
\renewcommand{\footrulewidth}{0pt} % Remove footer underlines
\setlength{\headheight}{13.6pt} % Customize the height of the header
\usepackage{listings}

\pgfplotsset{compat=1.10}
\numberwithin{equation}{section} % Number equations within sections (i.e. 1.1, 1.2, 2.1, 2.2 instead of 1, 2, 3, 4)
\numberwithin{figure}{section} % Number figures within sections (i.e. 1.1, 1.2, 2.1, 2.2 instead of 1, 2, 3, 4)
\numberwithin{table}{section} % Number tables within sections (i.e. 1.1, 1.2, 2.1, 2.2 instead of 1, 2, 3, 4)

\setlength\parindent{0pt} % Removes all indentation from paragraphs - comment this line for an assignment with lots of text

%----------------------------------------------------------------------------------------
%   TITLE SECTION
%----------------------------------------------------------------------------------------

\newcommand{\horrule}[1]{\rule{\linewidth}{#1}} % Create horizontal rule command with 1 argument of height

\title{ 
    \normalfont \normalsize 
    \textsc{Politechnika Warszawska} \\ [25pt] % Your university, school and/or department name(s)
    \horrule{0.5pt} \\[0.4cm] % Thin top horizontal rule
    \huge Obliczenia rozproszone w klastrach i gridach: Projekt dodatkowy\\ % The assignment title
    \horrule{2pt} \\[0.5cm] % Thick bottom horizontal rule
    System zarządzania klastrem StormMQ. Historia, budowa, funkcjonalność, specyficzne cechy.
}

\author{Mateusz Starzycki} % Your name

\date{\normalsize\today} % Today's date or a custom date

\begin{document}
\parskip 5 mm

\maketitle % Print the title

%----------------------------------------------------------------------------------------
%   PROBLEM 1
%----------------------------------------------------------------------------------------

\newpage

\section{Wstęp}

StormMQ jest to system kolejkowania wiadomości zwany inaczej brokerem albo oprogramowaniem pośredniczącym dostarczającym wiadomości do 
komputerów w klastrze.

Klaster - systemy wielokomputerowe składające się z wielu maszyn powiązanych ze sobą fizycznie i logicznie

Oprogramowanie pośredniczące dostarczające wiadomości (message oriented middleware, message broker) - oprogramowanie umożliwiające 
traktowanie zadań jako wiadomości, pośredniczy ono przy wymianie oraz dostarczeniu komunikatów do odpowiednich indywidualnych serwerów.
Oprogramowanie to tłumaczy także protokół nadawcy na protokół odbiorcy. Jest to wzorzec projektowy który ma na celu:
\begin{itemize}
    \item Przekazanie wiadomości do 1 lub więcej odbiorców.
    \item Przekształcenie wiadomości do alternatywnej reprezentacji.
    \item Agregację oraz enkapsulację a także połaczenie danych z powrotem w całość.
    \item Interakcję z zewnętrznymi serwisami składowania oraz wzbogacania wiadomości.
    \item Komunikacja z serwisami udostępniającymi dostęp do sieci.
    \item Obsługa błędów oraz wydarzeń systemowych.
    \item Dostarczenie wiadomości na podstawie treści i nagłowka na zasadzie producent, subskrybent.
\end{itemize}

\section{Historia}

\subsection{Wydarzenia}

\begin{itemize}
    \item 1900 - telegramy wysyłane metodą zapisz i przekaż
    \item 1948 - elektroniczne maszyny telegraficzne (np. Plan 55-A)
    \item 1964 - IBM system pierszy switching wiadomości
    \item 1965 - Pierwsza poczta elektroniczna
    \item 1971 - IBM TCAM pierwsze rozwiązanie kolejkowania wiadomości, funkcjonowało do 1990
    \item 1980 - Rozwój protokołu SMTP powstaje system Tibico w sektorze giełdowym
    \item 1992 - IBM uruchamia MQSeries obecnie WebshareMQ
    \item 2001 - Firma Sun wypuszcza Java JMS 
    \item 2006 - Powstaje AMPQP - grupa stworzona przez banki inwestycyjne
    \item 2009 - Rozwiązania w chmurze umożlwiają wzrost popularności systemu Storm MQ
\end{itemize}

\subsection{Założenie Firmy}

StormMQ zostało założone w roku 2009 przez Eamon Walshe oraz Raphaela ‘Raph’ Cohn oraz kilku inżynierów
pracujących wcześniej w firmie British Telecommunication.
Dostrzegli oni potrzebę na szybką w konfiguracji oraz łatwej we wdrożeniu kolejki obsługi
wiadomości. W grupie BT zajmowali się oni poprawą infrastruktury oraz redukcją kosztów projektów.
Bardzo często ich praca sprowadzała się także do obserwacji przyczyn które powodowały że projekty kończyły się
niepowodzeniem. Jedną z częstszych przyczyn okazały się problemy z klastrami; serwery były bardzo kosztowne w utrzymaniu,
wdrożeniu oraz same urządzenia wraz z miejscem ich przechowywania podnosiły cenę takich rozwiązań.
Próbowano różne podejścia w celu zniwelowania tego problemu. Modele płatności za wirtualne serwery sprawiły jednak
że i ta opcja była bardzo droga. 

Zidentyfikowana została zatem potrzeba innego podejścia do problemu - zarządzanie dynamicznymi zasobami zamiast kupowania
wirtualnych serwerów. W tym celu stworzony został system StormMQ. Jest on oparty o protokół AMQP oraz
umożliwia przesyłanie wiadomości pomiędzy serwisami na zasadzie producent, subskrybent.
W przypadku zarządzania dynamicznymi zasobami istnieje możliwość bardzo prostego odłączania oraz przyłączania zasobów.
Jedynym problemam jest rozwiązanie komunikacji która zwykle odbywała się przy pomocy skomplikowanych systemów opartych
na bazie danych. W celu ułatwienia stworzony został system pośredni StormMQ, oparty o "chmurę", gotowy do użycia 
łatwy do skonfigurowania.

\subsection{Problemy ze wcześniejszymi technologiami}

Rozdział ten prezentuje największe problemy we wcześniejszych technologiach stosowanych do
zarządzania komunikacją w klastrze:

\subsection{Wirtualne serwery}

Bardzo drogie w większych ilościach, ciężkie w skalowaniu, dużo cięższe w skalowaniu w "dół".
znalezienia oraz naprawy.


\subsection{Serwery}

Potrzeba zarządzania, przechowania, ułożenia protokołu komunikacji. Drogie w utrzymaniu.
Bardzo ciężkie we wdrożeniu. Infrastruktura zabiera tygodnie zanim urządzenia są u klienta,
do tego przy wdrożeniu komunikacji potrzebne są znaczne nakłady IT w celu skonfigurowania serwerów.
Ceny konsulatantów są bardzo duże, odpowiedzilaność znajduje się po stronie użytkownika.

\subsection{Bazy danych}

Rozwiązania oparte o bazy danych są bardzo skomplikowane, konsultanci są bardzo drodzy,
odpowiedzialność znajduje się po stronie użytkownika, skalowalność jest bardzo ciężka bez
dużych nakładów finansowych.

\section{Cechy szczególne}

StormMQ powstał jako łatwy do konfiguracji system kolejkowania wiadomości, jest to zdecydowanie
najistotniejsza oraz najbardziej charakterystyczna cecha systemu.

Kolejną nietypową cechą jest fakt iż jest on oparty o 'chmurę' cała infrastruktóra znajduje się
po stronie firmy StormMQ.

Dodatkowo jest on darmowy do pięciu połączeń, umożliwia to łatwe testowanie systemu.

Ważną cechą jest także model płatności - nieograniczone jest ilość przesłanych wiadomości a ilość
danych wykupionych, jeśli wiadomości są przetwarzene odpowiednio szybko koszty eksploatacji systemu
są niskie.

Dobra skalowalność to kolejna cecha systemu, pozwala on na osiągnięcie transferu nawet do 850000 wiadomości
na sekundę.

Protokół działa na zasadzie zapisz i przekaż, niebezpieczeństwo utracenia wiadomości jest minimalne.

System umożliwia na dynamiczne zarządzani subskrybentami.

Bezpieczeństwo wiadomości - domyślnie kolejka działa w trybie enkrypcji danych klienta.

\section{Największe problemy w przeszłości oraz ich rozwiązania}

W przypadku problemów technicznych z częścią serwerów możliwe jest całkowita
awaria systemu. StormMQ star się nie dopuścić do problemy z komunikacją. Twórcy
podczas swojej kariery zauważyli iż zamiast automatycznego odzyskiwania systemu
łatwiej jest niedopuścić do problemów z  komunikacją lub awariami serwerów.
W tym celu wprowadzili proces wieloktrotnego łączenia komputerów redundantnymi połączeniami.
Dodatkowo serwery mają taki sam, lub bardzo zbliżony hardware oraz system operacyjny.
Procedura restartu serwera w przypadku awarii w systemie StormMQ to załadowanie całkowitego
obrazu systemu oraz programu z zewnętrznej pamięci.

Problemy zależności zachowań programu od systemu operacyjnego. Wszystkie systemy w firmie
StormMQ mają to samo jądro systemu. Dodatkowo firma stara się mieć wszystkie źródła do systemów oraz
oprogramowania które instalują. W przyszłości planują oni nawet utworzenie własnej dystrybucji linuxa.

Problemy ludzkie, złe połączenia, zamiany serwerów, zmiany kabli. Firma StormMQ próbuje stosować procesy
uniemożliwiające lub minimalizującę impakt ludzi na system.

Problem rozdzielonego umysłu - część komunikacji umiera, zostaje wybrany nowy "master", sieć znów zaczyna działać
i dwa różne serwery sądzą że są serwerami nadrzędnymi. StormMQ rozwiązuje ten problem próbując uniknąć go. Podczas
wielu lat testowania założyciele próbowali rozwiązać problem różnymi metodami. Koniec końców zdecydowali iż, najłatwiejsze
i najtańsze podejście to próba niedpouszczenia do problemu. Koszty manualnego odwracania awarii są dużo tańsze niż próba
rozwinięcia automatycznego systemu.

\section{Zastosowania}

W rozdziale tym znajdują się przykładowe zastosowania dla systemu.

\begin{itemize}
    \item{Telekomunikacja} Stosowane do udostępnienia dodatkowych serwisów w telekomunikacji.
    \item{Finanse} Stosowane do zarządzania klientami oraz płatnościami.
    \item{Giełda} Stosowane do przewidywania przyszłych zachowań giełdowych oraz handlowania na akcjami.
    \item{Logistyka} Stosowane do zarządzania danymi diagnostycznymi pojzadów.
\end{itemize}

\section{Przyszłość}

Przyszłość systemu StormMQ maluje się w różnych barwach.
Jest to jeden z nietypowych systemów z dość niskim oraz sprawiedliwym systemem płatności,
jednocześnie podejście użytkowników stale zmienia się utrudniając jakiekolwiek zarobki.
Użytkownicy coraz bardziej zawyżają swoje wymagania co do jakości produktów oraz zaniżają
ceny, oczekują oni cen porównywalnych do aplikacji w "apple store".

Z drugiej strony niekożystne decyzje finansowe sprawiły iż firma straciła spore źródło dochodu.
Zaangażowanie w kontrakt z jednym dużym klientem zamiast wieloma małymi podważyło stabilność
finansową firmy.

Wraz z rozwojem własnej dystrybucji systemu Linux potencjalnie system może stać się jednak dużo bardziej stabilny
umożliwiając pozyskanie nowych klientów oraz podwyżenie niezwodności systemu.

\end{document}
