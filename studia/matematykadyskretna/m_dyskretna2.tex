%%%%%%%%%%%%%%%%%%%%%%%%%%%%%%%%%%%%%%%%%
% Short Sectioned Assignment
% LaTeX Template
% Version 1.0 (5/5/12)
%
% This template has been downloaded from:
% http://www.LaTeXTemplates.com
%
% Original author:
% Frits Wenneker (http://www.howtotex.com)
%
% License:
% CC BY-NC-SA 3.0 (http://creativecommons.org/licenses/by-nc-sa/3.0/)
%
%%%%%%%%%%%%%%%%%%%%%%%%%%%%%%%%%%%%%%%%%

%----------------------------------------------------------------------------------------
%   PACKAGES AND OTHER DOCUMENT CONFIGURATIONS
%----------------------------------------------------------------------------------------

\documentclass[paper=a4, fontsize=11pt]{scrartcl} % A4 paper and 11pt font size

\usepackage[T1]{fontenc} % Use 10-bit encoding that has 256 glyphs
\usepackage{polski}
\usepackage[utf8]{inputenc}
\usepackage[polish]{babel}
\usepackage{fontspec}
\usepackage{pythontex}
\usepackage{xunicode}
\usepackage{xltxtra}
\usepackage{babel}
\usepackage{tikz}
\usetikzlibrary{shapes,backgrounds}

\usepackage{amsmath,amsfonts,amsthm} % Math packages
\usepackage{pgfplots} % Math packages
\usepackage{lipsum} % Used for inserting dummy 'Lorem ipsum' text into the template
\usepackage{enumerate}
\usepackage{sectsty} % Allows customizing section commands
\usepackage{fancyhdr} % Custom headers and footers
\pagestyle{fancyplain} % Makes all pages in the document conform to the custom headers and footers
\fancyhead{} % No page header - if you want one, create it in the same way as the footers below
\fancyfoot[L]{} % Empty left footer
\fancyfoot[C]{} % Empty center footer
\fancyfoot[R]{\thepage} % Page numbering for right footer
\renewcommand{\headrulewidth}{0pt} % Remove header underlines
\renewcommand{\footrulewidth}{0pt} % Remove footer underlines
\setlength{\headheight}{13.6pt} % Customize the height of the header
\usepackage{listings}

\pgfplotsset{compat=1.10}
\numberwithin{equation}{section} % Number equations within sections (i.e. 1.1, 1.2, 2.1, 2.2 instead of 1, 2, 3, 4)
\numberwithin{figure}{section} % Number figures within sections (i.e. 1.1, 1.2, 2.1, 2.2 instead of 1, 2, 3, 4)
\numberwithin{table}{section} % Number tables within sections (i.e. 1.1, 1.2, 2.1, 2.2 instead of 1, 2, 3, 4)

\setlength\parindent{0pt} % Removes all indentation from paragraphs - comment this line for an assignment with lots of text

%----------------------------------------------------------------------------------------
%   TITLE SECTION
%----------------------------------------------------------------------------------------

\newcommand{\horrule}[1]{\rule{\linewidth}{#1}} % Create horizontal rule command with 1 argument of height

\title{ 
    \normalfont \normalsize 
    \textsc{Politechnika Warszawska} \\ [25pt] % Your university, school and/or department name(s)
    \horrule{0.5pt} \\[0.4cm] % Thin top horizontal rule
    \huge Matematyka Dyskretna zestaw zadań 2\\ % The assignment title
    \horrule{2pt} \\[0.5cm] % Thick bottom horizontal rule
}

\author{Mateusz Starzycki} % Your name

\date{\normalsize\today} % Today's date or a custom date

\usepackage{xcolor,colortbl}

\newcommand{\mc}[2]{\multicolumn{#1}{c}{#2}}
\definecolor{Gray}{gray}{0.85}
\definecolor{LightCyan}{rgb}{0.88,1,1}

\newcolumntype{a}{>{\columncolor{Gray}}c}
\newcolumntype{b}{>{\columncolor{white}}c}

\begin{document}

\maketitle % Print the title

%----------------------------------------------------------------------------------------
%   PROBLEM 1
%----------------------------------------------------------------------------------------

\newpage

\section{Zadanie}

W celu policzenia ile jest liczb niewiekszych od 2000 nie podzielnych przez 2,6,8 nalezy zauwazyc iz
każda liczba podzielna przez 8 będzie podzielna przez 2, innymi słowy, odejmując wszystkie liczby podzielne
przez dwa, odejmiemy także wszystkie podzielne przez 8. Możemy więc sprowadźić to samo zadanie do problemu
ile jest liczb nie podzielnych przez 6 lub 2. Każda liczba podzielna przez 6 dzieli się przez dwa.
Z podzielności przez 6 oraz 8 wynika zatem natychmiast podzielność przez 2.

Jest 2000 liczb niemniejszych od 2000.

Z czego 1000 dzieli się przez 2.

Zostaje 1000 liczb nie podzielnych przez 2, i jest to wynik zadania.

\section{Zadanie}

W celu policzenia ilu kibiców dostało uraz głowy należy zauważyić iż jesli każdy dostał 
jakiegoś urazu, mamy do czyenienia z przecięciem trzech grup, przy czym \(A\)- grupa
odnosząca uraz głowy, \(B\)- grupa odnosząca uraz ręki \(C\) - Grupa odnoszaca uraz nogi.

\begin{equation}
\nonumber
\mid B \mid = 15
\end{equation}
\begin{equation}
\nonumber
\mid C \mid = 17
\end{equation}
\begin{equation}
\nonumber
\mid A \cap B \mid = 8
\end{equation}
\begin{equation}
\nonumber
\mid A \cap C \mid = 11
\end{equation}
\begin{equation}
\nonumber
\mid B \cap C \mid = 10
\end{equation}
\begin{equation}
\nonumber
\mid A \cap B \cap C \mid = 5
\end{equation}
\def\firstcircle{(0,0) circle (1.5cm)}
\def\secondcircle{(60:2cm) circle (1.5cm)}
\def\thirdcircle{(0:2cm) circle (1.5cm)}
\begin{tikzpicture}
    \begin{scope}[shift={(3cm,-5cm)}, fill opacity=0.5]
        \fill[red] \firstcircle;
        \fill[green] \secondcircle;
        \fill[blue] \thirdcircle;
        \draw \firstcircle node[below] {$A$};
        \draw \secondcircle node [above] {$B$};
        \draw \thirdcircle node [below] {$C$};
    \end{scope}
\end{tikzpicture}

W celu obliczenia mocy zbioru A możemy posłużyć się wzorem na obliczenia mocy uniwersum na podstawie mocy zbioru:

\begin{equation}
  \mid U \mid = \mid A \mid + \mid B \mid + \mid C \mid - \mid A \cap B \mid - \mid A \cap C \mid - \mid B \cap C \mid + \mid A \cap B \cap C \mid
\end{equation}

Wystarczy przekształcić ten wzór w następujący sposób:

\begin{equation}
  \nonumber
  \mid A \mid = \mid U \mid - \mid B \mid - \mid C \mid + \mid A \cap B \mid +
\end{equation}
\begin{equation}
  \nonumber
  + \mid A \cap C \mid + \mid B \cap C \mid - \mid A \cap B \cap C \mid = 
\end{equation}
\begin{equation}
  = 40 - 15 - 17 + 8 + 11 + 10 - 5 = 32
\end{equation}


\section{Zadanie}

Zapiszmy ile wynosi ciąg tworzący ciągu:

\begin{equation}
  a_n = n + 1
\end{equation}

Podstawiając do definicji otrzymujemy:

\begin{equation}
  \sum_{n=0}^{\infty} (n + 1)* x^n = \sum_{n=0}^{\infty} \frac{d}{dx}x^{n+1}=\frac{d}{dx}\sum_{n=0}^{\infty} x^{n+1} = \frac{d}{dx} * \frac{x}{1 - x} = \frac{1}{(1-x)^2}
\end{equation}

W przedostatnim przejściu korzystamy z tego iż szereg tworzący samego ciągu jedynek wynosi:

\begin{equation}
  G(x) = \sum_{n=0}^{\infty}x^n = \frac{1}{1 - x} 
\end{equation}


W przypadku drugiego ciągu zaczynamy od:

\begin{equation}
  a_n = n * 5 ^n 
\end{equation}

Znów zaczynając od definicji mamy:

\begin{equation}
  \sum_{n=0}^{\infty} (n * 5^n )* x^n
\end{equation}

Podkładając \(y = 5x\) mamy, następnie stosujemy przesunięcie:

\begin{equation}
  \sum_{n=0}^{\infty} (n)* y^n = y^{-1} * \sum_{n=-1}^{\infty} (n + 1)* y^n = \frac{1}{y} * \frac{1}{(1-y)^2}= \frac{1}{y(1-y)^2}= \frac{1}{5x(1-5x)^2}
\end{equation}

\section{Zadanie}

Wzór rekurencyjny na ciąg wynosi:

\begin{equation}
  a_n = a_{n-1} * 1.6 + 0.4
\end{equation}

Warunek początkowy zaś:

\begin{equation}
  a_0 = 1000
\end{equation}

Doliczyć można łatwo \(a_1\):
\begin{equation}
  a_1 = a_0 * 1.6 + 400 = 2000
\end{equation}

Korzystając z twierdzania numer 3 z wykładu 8 wiemy, iż wzór jawny na \(a_n\) wynosi:

\begin{equation}
  a_n = c * 1.6^n + d
\end{equation}

Podstawiając do wzoru pierwszy oraz drugi wyraz, otrzymujemy:
\begin{equation}
  c + d = 1000
\end{equation}

Oraz

\begin{equation}
  1.6 * c + d = 2000
\end{equation}

Odejmując stronami otrzymujemy:

\begin{equation}
  0.6 * c = 1000
\end{equation}

Wzór jawny na zysk wynosi zatem:

\begin{equation}
  a_n = 1666 \frac{2}{3} * 1.6^n - 666\frac{2}{3} = 1000
\end{equation}


\end{document}
