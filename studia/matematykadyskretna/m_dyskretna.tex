%%%%%%%%%%%%%%%%%%%%%%%%%%%%%%%%%%%%%%%%%
% Short Sectioned Assignment
% LaTeX Template
% Version 1.0 (5/5/12)
%
% This template has been downloaded from:
% http://www.LaTeXTemplates.com
%
% Original author:
% Frits Wenneker (http://www.howtotex.com)
%
% License:
% CC BY-NC-SA 3.0 (http://creativecommons.org/licenses/by-nc-sa/3.0/)
%
%%%%%%%%%%%%%%%%%%%%%%%%%%%%%%%%%%%%%%%%%

%----------------------------------------------------------------------------------------
%   PACKAGES AND OTHER DOCUMENT CONFIGURATIONS
%----------------------------------------------------------------------------------------

\documentclass[paper=a4, fontsize=11pt]{scrartcl} % A4 paper and 11pt font size

\usepackage[T1]{fontenc} % Use 10-bit encoding that has 256 glyphs
\usepackage{polski}
\usepackage[utf8]{inputenc}
\usepackage[polish]{babel}
\usepackage{fontspec}
\usepackage{pythontex}
\usepackage{xunicode}
\usepackage{xltxtra}
\usepackage{babel}

\usepackage{amsmath,amsfonts,amsthm} % Math packages
\usepackage{pgfplots} % Math packages
\usepackage{lipsum} % Used for inserting dummy 'Lorem ipsum' text into the template
\usepackage{enumerate}
\usepackage{sectsty} % Allows customizing section commands
\usepackage{fancyhdr} % Custom headers and footers
\pagestyle{fancyplain} % Makes all pages in the document conform to the custom headers and footers
\fancyhead{} % No page header - if you want one, create it in the same way as the footers below
\fancyfoot[L]{} % Empty left footer
\fancyfoot[C]{} % Empty center footer
\fancyfoot[R]{\thepage} % Page numbering for right footer
\renewcommand{\headrulewidth}{0pt} % Remove header underlines
\renewcommand{\footrulewidth}{0pt} % Remove footer underlines
\setlength{\headheight}{13.6pt} % Customize the height of the header
\usepackage{listings}

\pgfplotsset{compat=1.10}
\numberwithin{equation}{section} % Number equations within sections (i.e. 1.1, 1.2, 2.1, 2.2 instead of 1, 2, 3, 4)
\numberwithin{figure}{section} % Number figures within sections (i.e. 1.1, 1.2, 2.1, 2.2 instead of 1, 2, 3, 4)
\numberwithin{table}{section} % Number tables within sections (i.e. 1.1, 1.2, 2.1, 2.2 instead of 1, 2, 3, 4)

\setlength\parindent{0pt} % Removes all indentation from paragraphs - comment this line for an assignment with lots of text

%----------------------------------------------------------------------------------------
%   TITLE SECTION
%----------------------------------------------------------------------------------------

\newcommand{\horrule}[1]{\rule{\linewidth}{#1}} % Create horizontal rule command with 1 argument of height

\title{ 
    \normalfont \normalsize 
    \textsc{Politechnika Warszawska} \\ [25pt] % Your university, school and/or department name(s)
    \horrule{0.5pt} \\[0.4cm] % Thin top horizontal rule
    \huge Matematyka Dyskretna zestaw zadań 1\\ % The assignment title
    \horrule{2pt} \\[0.5cm] % Thick bottom horizontal rule
}

\author{Mateusz Starzycki} % Your name

\date{\normalsize\today} % Today's date or a custom date

\usepackage{xcolor,colortbl}

\newcommand{\mc}[2]{\multicolumn{#1}{c}{#2}}
\definecolor{Gray}{gray}{0.85}
\definecolor{LightCyan}{rgb}{0.88,1,1}

\newcolumntype{a}{>{\columncolor{Gray}}c}
\newcolumntype{b}{>{\columncolor{white}}c}

\begin{document}

\maketitle % Print the title

%----------------------------------------------------------------------------------------
%   PROBLEM 1
%----------------------------------------------------------------------------------------

\newpage

\section{Zadanie}

Zadanie można podzielić na dwie części.

\subsection{Zliczenie wszystkich kombinacji}

Część pierwsza zadania to zliczenie wszystkich możliwych kombinacji,
rozruźniamy więc treść zadania omijając drugą jego część (3 jedynki stanowiące kolejne wyrazy)
Mamy zatem doczynienia z problemem w których musimy zliczyć wszystki możliwe permutacje z powtórzeniami
zbioru o mocy 7, w którym występują 3 powtórzenia jednego elementu (jedynek) oraz 4 powtórzenia drugiego.

Do takiego probelmu nalezy zastosować wzór:

\begin{equation}
  P_n^{n_1...n_k} = \frac{n!}{n_1!*...n_k!}
\end{equation}

Podstawiając wartości do zbioru \(n = 7, n_1 = 3, n_2 = 4\)otrzymujemy:

\begin{equation}
  P_n^{n_1...n_k} = \frac{7!}{3!*4!}
\end{equation}

\subsection{Usunięcie wyrazów z potrójnymi jedynkami}

Kolejną częścią zadania jest usunięcie ze zliczonych permutacji tych, w których trzy jedynki występują koło siebie.
Ponieważ jest dostępnych do losowania jedynie właśnie tyle jedynek, wszystkie muszą występować koło siebie.
A co za tym idzie muszą one być jako blok na którejś z pozycji, albo rozpoczynający się od 1, od 2 albo od 4 pozycji,
przedstwia to poniższy schemat:

\begin{table}[h]
\centering
\begin{tabular}{| a | a | a | c | c | c | c | }
\hline
  1 & 1 & 1 & 0 & 0 & 0 & 0 \\ 
\hline
\end{tabular}
\end{table}
\begin{table}[h]
\centering
\begin{tabular}{| c | a | a | a | c | c | c | }
\hline
  0 & 1 & 1 & 1 & 0 & 0 & 0 \\ 
\hline
\end{tabular}
\end{table}

\begin{equation}
  \nonumber
  ...
\end{equation}
\begin{table}[h]
\centering
\begin{tabular}{| c | c | c | c | a | a | a | }
\hline
  0 & 0 & 0 & 0 & 1 & 1 & 1 \\ 
\hline
\end{tabular}
\end{table}

Istnieje dokładnie 5 takich permutacji, w związku z czym rozwiązanie zadania to:

\begin{equation}
  \frac{7!}{3!*4!} - 5
\end{equation}
\newpage
\section{Zadanie}

W zadaniu tym mamy sytuację, w której z nieograniczonego źródła wybieramy 20 przedmiotów.
Przedmioty te są nierozróżnialne. Mamy zatem zagadnienie związane z ilością sposobów
sumowania 3 liczb do pewnej sumy (20) gdzie każda z liczb określa ilość danego rodzaju piwa.
Wzór na ilość sposobów sumowania liczb do zadanej to:

\begin{equation}
  { n + k - 1 \choose n - 1 }
\end{equation}

Gdzie \(k\) to liczba do której sumujemy a \(n\) to ilość składników nieujemnych.

\subsection{Podpunkt a}

W tej części zadania pozbawieni zostaliśmy wyboru 6 butelek piwa, wkładamy je zatem
najpierw do koszyka, zostajemy w sytuacji prawie identycznej, jedyna różnica polega na tym
iż zostało nam do wyboru 14 butelek piwa, i ewentualny napój na dzień następny
W związku z tym w przypadku tym nasze \(k\) wynosi 14 a ilość składników nadal wynosi \(n = 3\). 
Podkładając zatem do wzoru mamy:

\begin{equation}
  { n + k - 1 \choose n - 1 } = {14 + 3 - 1 \choose 3 - 1} = { 16 \choose 2 } = \frac{16!}{2! * (14!)}
\end{equation}

\subsection{Podpunkt b}

W podpunkcie b sytuacja powtarza się, tym razem zostajemy jednak pozbawieni nie tylko wyboru 8 piw,
które weźmiemy najpierw i zostanie nam jedynie 12 do wyboru, pozbawieni zostajemy także możliwości
wybierania dodatkowych piw rodzaju B. Co za tym idzie zmniejszona zostaje także ilość składników.
Przeformułować możemy zatem zadanie na pytanie "Na ile sposobów możemy z pośród 2 rodzajów piwa wybrać
12 butelek". Odpowiedź to oczywiście 13. Możemy bowiem wziąć dokładnie 0 piw 1 rodzaju, dokładnie 1 ... itd
aż do dokładnie 12.
\newpage
\section{Zadanie}

W celu wyznaczenia sumy:

\begin{equation}
  \sum_{k=1}^{200}{200 \choose k} * k * (-3)^k
\end{equation}

Przekształcimy ją nieco:

\begin{equation}
  \sum_{k=1}^{200}{200 \choose k} * k * (-3)^k = -3 * \sum_{k=1}^{200}{200 \choose k} * k * (-3)^{k - 1}
\end{equation}

Zastosować można zatem wzór:

\begin{equation}
  n*(x+1)^{n-1} = \sum_{k=1}^{n} {n \choose k} * k * (x)^k
\end{equation}

Gdzie \(n\) to 200 natomiast \(x\) to -3

Podkładając dostajemy:

\begin{equation}
  -3 * \sum_{k=1}^{200}{200 \choose k} * k * (-3)^{k - 1} = -3 * 200 * (-3 + 1)^{200 - 1} = -600 * (-2)^{199}
\end{equation}



\newpage
\section{Zadanie}

\begin{equation}
  \nonumber
  S(6,4) = S(5,3) + 4 * S(5,4) = 25 + 4 * 10 = 65
\end{equation}

\begin{equation}
  \nonumber
  S(5,3) = S(4,2) + 3 * S(4,3) = 7 + 3 * 6 = 25 
\end{equation}

\begin{equation}
  \nonumber
  S(4,2) = S(3,1) + 2 * S(3,2) = 1 + 2 * 3 = 7
\end{equation}

\begin{equation}
  \nonumber
  S(4,3) = S(3,2) + 3 * S(3,3) = 3 + 3 * 1 = 6 
\end{equation}

\begin{equation}
  \nonumber
  S(3,2) = S(2,1) + 2 * S(2,2) = 1 + 2 * 1 = 3 
\end{equation}

\begin{equation}
  \nonumber
  S(5,4) = S(4,3) + 4 * S(4,4) = 6 + 4 * 1 = 10
\end{equation}


\newpage
\section{Zadanie}
W zadaniu tym mamy do dyspozycji 5 krasnali (zakładam iż są one rozróżnialne).
Rozdzielić je mamy w 4 szufladach.

\subsection{Podpunkt a}
W tym podpunkcie szuflady są rozróżnialne, zatem metodą losowania może być przydzielenie każdemu z 
krasnali losu loteri od 1 do 4, oznaczać będzie to jego szufladę. Po powtórzeniu 5 krotnym, każdy z krasnali
będzie ulokowany.
Ponieważ występuje tutaj 5 etapów z 4 możliwościami wystarczy pomnożyć ilość możliwości na każdym z nich.
Ilość możliwych rozłożeń wynosi zatem:

\begin{equation}
  k^n = 4 ^ 5
\end{equation}

Gdzie \(k\) to ilość możliwości \(n\) ilość etapów losowania.

\subsection{Podpunkt b}

W przypadku tym nie mamy informacji o tym która szuflada jest która.
W związku z tym mamy przypadek rozdzielenia rozróżnialnych elementów
w nierozróżnialnych kategoriach,  skożystamy zatem z wzoru dla którego
nasze części są nierozrużnialne, a obiekty są rozróżnialne, wynosi on:

\begin{equation}
  \sum_{i=1}^{k}S(n,i)
\end{equation}

W naszym przypadku ilość naszych części \(k\) to 4 natomiast ilość obiektów \(n\) to 5.
Sumę tę rozwinąć zatem możemy do:

\begin{equation}
  \sum_{i=1}^{k}S(n,i) = S(5,1) + S(5,2) + S(5,3) + S(5,4)
\end{equation}
\begin{equation}
  S(5,1) = 1
\end{equation}
\begin{equation}
  \nonumber
  S(5,2) = S(4,1) + 2 * S(4,2) = 1 + (S(3,1) + 2 * S(3,2)) = 
\end{equation}
\begin{equation}
  1 + ( 1 + 2 * (S(2,1) + 2 * S(2,2))) = 1 + 2 * 7 = 15
\end{equation}
\begin{equation}
  \nonumber
  S(5,3) = S(4,2) + 3 * S(4,3) = 7 + 3 * S(4,3) = 
\end{equation}
\begin{equation}
  7 + 3 * (S(3,2) + 3 * S(3,3)) = 7 + 3 * ( 3 + 3 * 1) = 25
\end{equation}
\begin{equation}
  S(5,4) = S(4,3) + 4 * S(4,4) = 6 + 4 * 1 = 10
\end{equation}
Ostatecznie:
\begin{equation}
  \sum_{i=1}^{k}S(n,i) = S(5,1) + S(5,2) + S(5,3) + S(5,4) = 1 + 15 + 25 + 10 = 51 
\end{equation}

\end{document}
