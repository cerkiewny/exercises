%%%%%%%%%%%%%%%%%%%%%%%%%%%%%%%%%%%%%%%%%
% Short Sectioned Assignment
% LaTeX Template
% Version 1.0 (5/5/12)
%
% This template has been downloaded from:
% http://www.LaTeXTemplates.com
%
% Original author:
% Frits Wenneker (http://www.howtotex.com)
%
% License:
% CC BY-NC-SA 3.0 (http://creativecommons.org/licenses/by-nc-sa/3.0/)
%
%%%%%%%%%%%%%%%%%%%%%%%%%%%%%%%%%%%%%%%%%

%----------------------------------------------------------------------------------------
%   PACKAGES AND OTHER DOCUMENT CONFIGURATIONS
%----------------------------------------------------------------------------------------

\documentclass[paper=a4, fontsize=11pt]{scrartcl} % A4 paper and 11pt font size

\usepackage[T1]{fontenc} % Use 10-bit encoding that has 256 glyphs
\usepackage{polski}
\usepackage[utf8]{inputenc}
\usepackage[polish]{babel}
\usepackage{amsmath,amsfonts,amsthm} % Math packages
\usepackage{pgfplots} % Math packages
\usepackage{lipsum} % Used for inserting dummy 'Lorem ipsum' text into the template
\usepackage{enumerate}
\usepackage{sectsty} % Allows customizing section commands
\usepackage{fancyhdr} % Custom headers and footers
\pagestyle{fancyplain} % Makes all pages in the document conform to the custom headers and footers
\fancyhead{} % No page header - if you want one, create it in the same way as the footers below
\fancyfoot[L]{} % Empty left footer
\fancyfoot[C]{} % Empty center footer
\fancyfoot[R]{\thepage} % Page numbering for right footer
\renewcommand{\headrulewidth}{0pt} % Remove header underlines
\renewcommand{\footrulewidth}{0pt} % Remove footer underlines
\setlength{\headheight}{13.6pt} % Customize the height of the header
\usepackage{listings}

\pgfplotsset{compat=1.10}
\numberwithin{equation}{section} % Number equations within sections (i.e. 1.1, 1.2, 2.1, 2.2 instead of 1, 2, 3, 4)
\numberwithin{figure}{section} % Number figures within sections (i.e. 1.1, 1.2, 2.1, 2.2 instead of 1, 2, 3, 4)
\numberwithin{table}{section} % Number tables within sections (i.e. 1.1, 1.2, 2.1, 2.2 instead of 1, 2, 3, 4)

\setlength\parindent{0pt} % Removes all indentation from paragraphs - comment this line for an assignment with lots of text

%----------------------------------------------------------------------------------------
%   TITLE SECTION
%----------------------------------------------------------------------------------------

\newcommand{\horrule}[1]{\rule{\linewidth}{#1}} % Create horizontal rule command with 1 argument of height

\title{ 
    \normalfont \normalsize 
    \textsc{Politechnika Warszawska} \\ [25pt] % Your university, school and/or department name(s)
    \horrule{0.5pt} \\[0.4cm] % Thin top horizontal rule
    \huge Systemy Czasu Rzeczywistego zadanie 1: Sieci AS-i\\ % The assignment title
    \horrule{2pt} \\[0.5cm] % Thick bottom horizontal rule
}

\author{Mateusz Starzycki} % Your name

\date{\normalsize\today} % Today's date or a custom date

\begin{document}

\maketitle % Print the title

%----------------------------------------------------------------------------------------
%   PROBLEM 1
%----------------------------------------------------------------------------------------

\newpage

\section{Zadanie 1.1}

Czas dostępu od urządzenia jest to parametr określający daną sieć.
Zależy on od warstwy fizycznej oraz danych sieci a także bardzo często (również w przypadku sieci as-i) od ilości urządzeń.

Czasami charakteryzującymi dostęp do urządzenia są czas minimalny oraz maksymalny \(t_{min}\) oraz \(t_{max}\).
Rozważyć można dwa przypadki, pierwszy - optymistyczny w którym urządzenie typu slave ma zostać wywołane przez urządzenie
typu master jako następne w kolejności. W tym wypadku załóżmy że wywoływanie rozpocznie się natychmiast. Urządzenie master
musi wysłać zatem 14 bitów pakietu wywołującego, następnie następuję pauza typu master o długości 3-10 bitów.
Ponieważ zajmujemy się minimalnym czasem dostępu, załóżmy zatem iż pauza trwa 3 bity. Następnym etapem jest odpowiedź
przez urządzenie typu slave, transmisja zajmuję 7 bitów i co za tym idzie cały transfer (nie liczę pauzy typu slave gdyż w tym czasię
urządzenie master może rozpocząć już analizę danych) wynosi 24 bity, co przy prędkości 167 kb/s wynosi:

\[1s / 167000 * 24 = 144 \mu s\]

Dodatkowo możliwe jest nieznaczne przyspieszenie spowodowane przyspieszonymi zboczami podkładając zatem przesłane przez mastera 14 i sleva
7 bitów otrzymujemy przyspieszenie maksymalne zgodne z tolerancją wynosi -0.5 \(\mu s\) zyskujemy zatem dodatkową \(1 \mu s\).
Czyli minimalny czas dostępu wynosi \(143 \mu s\)


Warto zauważyć iż czas minimalny nie zależy od ilości urządzeń w sieci, należy jednak pamiętać że zależy prawdopodobieństwo jego wystąpienia.

Zakładając natomiast najbardziej pesymistyczny przypadek możemy wyliczyć maksymalny czas dostępu. Zakładając że transfer z urządzeniem
właśnie się zakończył musimy przeczekać na naszą kolej, urządzenie typu master wykona cykliczne odpytanie pozostałych urządzeń.
Zakładając że w sieci istnieje 9 innych urządzeń (oraz urządzenie zerowe) należy dodać zatem 10 cykli odpytania. Każdy z nich może zając
maksymalnie 31 bitów, należy także uwzględnić tym razem 1 bit stopu typu slave po każdej transmisji, co daje nam 32 bity na urządzenie.
Całkowity czas wynosi zatem:

\[1/ 167000 * 32 * 10 = 1,91 ms \]


Dodatkowym czynnikiem wpływającym na wydłużenie się tego czasu jest tolerancja długości sygnału. Każdy z 10 sygnałów może się wydłużyć o
tolerancje impulsów przy obu transferach, urządzenie typu master nadaje 14 bitów urządzenie typu slave natomiast 7.
Podkładamy więc obie te liczby do wzoru na maksymalne tolerowane opóźnienie:

\[(14+3\mu s) + 1\ = 18 \mu s]\]
\[(7+3\mu s) + 1\ = 11 \mu s]\]

Co dodane do całości powoduje dodatkowe opóźnienie o 290 \(\mu s\) całkowite opóźnienie wynosi zatem \(2.3 ms\).
Należy pamiętać jednak iż są to czasy przesyłu jednego transferu. W przypadku gdy urządzenie typu slave musi przesłać więcej niż 4 bity
informacji, czas dostępu znacznie się wydłuża.

W przypadku gdy urządzeń jest zaledwie 10, wersja standardu nie odgrywa roli w czasach dostępu.

\section{Zadanie 1.2}

W celu obliczenia ilości niezbędnych urządzeń należy przede wszystkim wyliczyć ile urządzeń typu slave
niezbędnych będzie do realizacji układu. Każde urządzenie typu slave może zostać połączone z 4 urządzeniami wejscia wyjscia.
Całkowita ilość urządzeń końcowych wynosi 1200 sensorów oraz 600 aktorów czyli 1200 urządzeń wejścia i 600 wyjścia. Do realizacji układy z taką ilością
niezbędne jest 300 urządzeń typu slave, gdyż możliwe jest podłączenie aż do 4 urządzeń wejścia i 4 wyjścia do każdego z nich. 

Zgodnie ze specyfikacją 2.0 urządzenie master może adresować do 31 urządzeń typu slave co za tym idzie niezbędnych będzie 10 urządzeń
typu master.
W przypadku gdy zastosowany zostanie standard 2.11 ich liczba będzie wynosić zaledwie 5, jedno z nich może adresować bowiem aż do 62 urządzeń
typu slave.


\section{Zadanie 1.3}


W celu pokrycia wszystkich urządzeń wyjścia potrzebujemy 200 urządzeń 4e3a. Do tego zostaje nam 400 urządzeń wejściowych które
należy umieścić w kolejnych 100 urządzeniach 4e. Koszt całkowity wynosi wtedy 3000 kredytów i jest taki sam jak przy zastosowaniu
urządzeń 4e4a oraz 4e. Należy zatem rozłożyć optymalnie 300 urządzeń w sieciach. Najlepszym ułożeniem jest ułożenie ich w 5 sieci
typu 2.11, co za tym idzie potrzebne jest 5 urządzeń master typu 2.11, tyle samo zasilaczy oraz 200 urządzeń 4e3a oraz 100 typu 4e.
Całkowity koszt wynosi zatem 4000.

\section{Zadanie 1.4}

\subsection{1 milisekunda}

W przypadku opóźnienia maksymalnego \(1ms\) ilość maksymalna urządzeń typu slave dla każdego urządzenia master wynosi 5.

W tym przypadku nie ma sensu korzystać z urządzeń master typu 2.11.

Urządzenia należy podzielić na jaknajmniej zbiorów sieci typu 2.0, co za tym idzie kluczowe jest umieszczenie urządzeń
wejścia w jaknajmniejszej ilości urządzeń typu slave. W tym celu zastosujemy jaknajwięcej urządzeń typu 4e3a (mają one maksymalną
ilość wyjść oraz są tańsze niż 4e4a a nie jest potrzebne aż tyle urżądzeń wyjściowych jest ich mniej niż 3/4). 
W celu pokrycia wszystkich urządzeń wyjścia potrzebujemy 200 urządzeń 4e3a. Do tego zostaje nam 400 urządzeń wejściowych które
należy umieścić w kolejnych 100 urządzeniach 4e. Co za tym idzie potrzebujemy 100 urządzeń typu master w wersjii 2.0 oraz 100 zasilaczy.
Całkowity koszt wyniesie zatem 20500 jednostek.

\subsection{5 milisekund}
W przypadku opóźnienia maksymalnego \(5ms\) ilość maksymalna urządzeń typu slave dla każdego urządzenia master wynosi 25.

W tym przypadku, tak samo jak i w poprzednim nie ma sensu wykorzystywać urządzeń typu master w wersji 2.11. Co za tym idzie
należy jednostki rozłożyć w więcej siecii, typy urządzeń slave zostają w dokładnie takiej samej ilośći, jest to bowiem optymalne
rozłożenie. W celu rozłożenia 300 urządzeń wystarczy nam tym razem 12 urządzeń typu master oraz 12 zasilaczy. Całkowity 
koszt takiej konstrukcji wynosi zatem 5100 jednostek.

\subsection{8 milisekund}
W przypadku opóźnienia maksymalnego \(8ms\) ilość maksymalna urządzeń typu slave dla każdego urządzenia master wynosi 40.

W tym przypadku możemy podzielić urządzenia na 6 sieci typu 2.11 oraz 2 typu 2.0. Całkowity koszt wyniesie wtedy 4550.

\subsection{12 milisekund}
W przypadku opóźnienia maksymalnego \(12ms\) ilość maksymalna urządzeń typu slave dla każdego urządzenia master wynosi 60.

W tym przypadku rozłożenie urządzeń typu master możemy ułożyć w 5 sieci typu 2.11 co za tym idzie koszt całkowity wyniesie 4200 jednostek.

\end{document}
