%%%%%%%%%%%%%%%%%%%%%%%%%%%%%%%%%%%%%%%%%
% Short Sectioned Assignment
% LaTeX Template
% Version 1.0 (5/5/12)
%
% This template has been downloaded from:
% http://www.LaTeXTemplates.com
%
% Original author:
% Frits Wenneker (http://www.howtotex.com)
%
% License:
% CC BY-NC-SA 3.0 (http://creativecommons.org/licenses/by-nc-sa/3.0/)
%
%%%%%%%%%%%%%%%%%%%%%%%%%%%%%%%%%%%%%%%%%

%----------------------------------------------------------------------------------------
%   PACKAGES AND OTHER DOCUMENT CONFIGURATIONS
%----------------------------------------------------------------------------------------

\documentclass[paper=a4, fontsize=11pt]{scrartcl} % A4 paper and 11pt font size

\usepackage[T1]{fontenc} % Use 10-bit encoding that has 256 glyphs
\usepackage{polski}
\usepackage[utf8]{inputenc}
\usepackage[polish]{babel}
\usepackage{amsmath,amsfonts,amsthm} % Math packages
\usepackage{pgfplots} % Math packages
\usepackage{lipsum} % Used for inserting dummy 'Lorem ipsum' text into the template
\usepackage{enumerate}
\usepackage{sectsty} % Allows customizing section commands
\usepackage{fancyhdr} % Custom headers and footers
\pagestyle{fancyplain} % Makes all pages in the document conform to the custom headers and footers
\fancyhead{} % No page header - if you want one, create it in the same way as the footers below
\fancyfoot[L]{} % Empty left footer
\fancyfoot[C]{} % Empty center footer
\fancyfoot[R]{\thepage} % Page numbering for right footer
\renewcommand{\headrulewidth}{0pt} % Remove header underlines
\renewcommand{\footrulewidth}{0pt} % Remove footer underlines
\setlength{\headheight}{13.6pt} % Customize the height of the header
\usepackage{listings}


\usetikzlibrary{external}
\tikzexternalize
\tikzset{external/force remake}

\pgfplotsset{compat=1.10}
\numberwithin{equation}{section} % Number equations within sections (i.e. 1.1, 1.2, 2.1, 2.2 instead of 1, 2, 3, 4)
\numberwithin{figure}{section} % Number figures within sections (i.e. 1.1, 1.2, 2.1, 2.2 instead of 1, 2, 3, 4)
\numberwithin{table}{section} % Number tables within sections (i.e. 1.1, 1.2, 2.1, 2.2 instead of 1, 2, 3, 4)

\setlength\parindent{0pt} % Removes all indentation from paragraphs - comment this line for an assignment with lots of text

%----------------------------------------------------------------------------------------
%   TITLE SECTION
%----------------------------------------------------------------------------------------

\newcommand{\horrule}[1]{\rule{\linewidth}{#1}} % Create horizontal rule command with 1 argument of height

\title{ 
    \normalfont \normalsize 
    \textsc{Politechnika Warszawska} \\ [25pt] % Your university, school and/or department name(s)
    \horrule{0.5pt} \\[0.4cm] % Thin top horizontal rule
    \huge Systemy Czasu Rzeczywistego zadanie 4: wstęp do projektu jednostki typu master\\ % The assignment title
    \horrule{2pt} \\[0.5cm] % Thick bottom horizontal rule
}

\author{Mateusz Starzycki} % Your name

\date{\normalsize\today} % Today's date or a custom date

\begin{document}

\maketitle % Print the title

%----------------------------------------------------------------------------------------
%   PROBLEM 1
%----------------------------------------------------------------------------------------

\newpage

\section{Zadanie 4.1}

W celu wyznacznia średnich czasów odstępów pomiędzy transmisjami napisany zostały dwa programy
jeden, działający w tle odbierał w czasie ciągłym dane z wirtualnego portu szeregowego, drugi
nadawał w sposób ciągły 500000 razy ten sam bajt, następnie sumował czas odstępu pomiędzy transmisjami.
Czas całkowity został podzielony przez
ilość wysłanych bajtów w celu oszacowania czasu średniego pomiędzy dwiema transmisjami.

Czas średni w przypadku gdy nie jest uruchomiona żadna aplikacja:

\[t_{avg}=3.78859996796\mu s\]

Czas średni w przypadku otwarcia 10 aplikacji:

\[t_{avg}=4.49301815033\mu s\]

Jak można było przypuszczać, zwiększenie używania zasobów systemowych powoduje spowolnienie programów
nie uznanych przez system operacyjny jako krytyczne czasowo. W związku z następującymi wywłaszczeniami 
na poczet aplikacji wideo oraz audio, program komunikujący się z portem szeregowym zostaje częściej wywłaszczony
co powoduje zwiększenie średniego czasu pomiędzy transmisjami.

\section{Zadanie 4.2}

W celu wyznaczenia czasów transmisji ciągu znaków napisane zostały dwa niezależnie działające programy.
Jeden był odpowiedzialny za wysyłanie ciągu znaków do wirtualnego portu szeregowego oraz mierzenie czasu który
upłynął od początku do końca transmisji, oraz programu ciągle odbierającego dane z wirtualnego portu.
Następnie na podstawie tych danych stworzone zostały 
histogramy czasów przesłania pomiędzy kolejnymi bajtami (obliczone dla czasu całkowitego podzielonego
na ilość bajtów w transmisjii).

Histogramy pokazują częstość występowania w danych grupach odstęp pomiędzy minimalnym a maksymalnym
czasem został podzielony na 100 grup następnie każdy z czasów został zakwalifikowany do odpowiednij grupy.

Dla uruchomienia programu bez działających aplikacji otrzymujemy następujący histogram:

\begin{tikzpicture}
  \begin{axis}[
       yticklabels={,,}
  ] 
    \addplot+ table [x, y, col sep=comma] {resseqnoapp.dat};
  \end{axis}
\end{tikzpicture}

Czasy minimalne oraz maksymalne to odpowiednio:
\[t_{min} = 3.81034283959 \mu s\]
\[t_{max} = 30.4877758026 \mu s\]

\begin{tikzpicture}
  \begin{axis}[
       yticklabels={,,}
  ] 
    \addplot+ table [x, y, col sep=comma] {resseqapps.dat};
  \end{axis}
\end{tikzpicture}

Czasy maksymalne oraz minimalne to odpowiednio:

\[t_{min} = 4.51889149122 \mu s\]
\[t_{max} = 4.97996807098 ms\]

Z histogramów wynika iż przy aplikacji działającej samodzielnie mamy dużo mniejszy rozrzut czasów pomiędzy poszczególnymi transmisjami.
Wiąże się to z tym, iż gdy w systemi nie działają żadne krytyczne czasowo programy (takie jak streaming obrazów lub dźwięku) aplikacja
może w niezakłócony sposób sekwencyjnie wykonywać program. W związku z tym iż system operacyjny zawsze musi cyklicznie przeprowadzać
introspekcję w celu identyfikacji zadań, w zależności od tego kiedy przypadnie cykl aktywności systemu operacyjnego, poszczególne 
wykonania w pętli mogą zostać nieznacznie opóźnione.

W przypadku dużej aktywności mamy dużo większe rozrzucenie wyników, niektóre z nich są 3 rzędy wielkości wyższe niż przy niezakłuconym wykonaniu
programu. Wiąże się to z faktem iż działające w tle aplikacje (głownie takie jak streaming mediów) mogą być aplikacjami krytycznymi czasowo.
Co za tym idzie mogą wywłaszczyć proces dokonujący obliczenia na procesorze i zająć procesor w celu natychmiastowego wykonania niezbędnych
czynności do niezakłóconego przebiegu aplikacji. Klasycznym przykładem może być na przykład wczytanie danych z karty sieciowej do bufora
filmu video w pamięci w przypadku zbliżania się do jego końca przez aplikację odtwarzającą. W związku z faktem iż systemy operacyjne takie jak
OSX (na którym testowano aplikację) przeznaczone są głównie dla użytkowników końcowych, promują one aplikację dostarczające użytkownikowi rozrywki.
Dużo ważniejsze jest dla nich obsłużenie komunikacji z urządzeniami wyjścia niż z portem szeregowym. W związku z czym nie możliwe jest zapewnienie
szybkiej komunikacji z portami zewnętrznymi.

\section{Wnioski}


W związku z tym iż procesy krytyczne mogą mięc znaczny wpływ na odstęp pomiędzy transmisjami systemy takie jak windows czy OSX nie są dobre
dla krytycznych czasowo zastosowań. Przypadkowe zabieranie zasobów przez systemy o wysokim priorytecie może wywłaszczyć proces który komunikuje się
z siecią modbus. Rozwiązaniem jest albo zmiana priorytetu wykonywanego programu, co jest rozwiązaniem niepewnym gdyż nie istnieje gwarancja
iż system operacyjny nie wywłaszczy go na poczet procesu systemowego, lub stosowanie systemów specjalnie dopasowanych do realizacja zadań
krytycznych czasowo.

\end{document}
