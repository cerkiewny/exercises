
%%%%%%%%%%%%%%%%%%%%%%%%%%%%%%%%%%%%%%%%%
% Short Sectioned Assignment
% LaTeX Template
% Version 1.0 (5/5/12)
%
% This template has been downloaded from:
% http://www.LaTeXTemplates.com
%
% Original author:
% Frits Wenneker (http://www.howtotex.com)
%
% License:
% CC BY-NC-SA 3.0 (http://creativecommons.org/licenses/by-nc-sa/3.0/)
%
%%%%%%%%%%%%%%%%%%%%%%%%%%%%%%%%%%%%%%%%%

%----------------------------------------------------------------------------------------
%   PACKAGES AND OTHER DOCUMENT CONFIGURATIONS
%----------------------------------------------------------------------------------------

\documentclass[paper=a4, fontsize=11pt]{scrartcl} % A4 paper and 11pt font size

\usepackage[T1]{fontenc} % Use 10-bit encoding that has 256 glyphs
\usepackage[english]{babel}
\usepackage{polski}
\usepackage[utf8]{inputenc}
\usepackage{amsmath,amsfonts,amsthm} % Math packages
\usepackage{pgfplots} % Math packages
\usepackage{lipsum} % Used for inserting dummy 'Lorem ipsum' text into the template
\usepackage{enumerate}
\usepackage{sectsty} % Allows customizing section commands
\pgfplotsset{compat=1.11}
\allsectionsfont{\centering \normalfont\scshape} % Make all sections centered, the default font and small caps
\usepackage{fancyhdr} % Custom headers and footers
\pagestyle{fancyplain} % Makes all pages in the document conform to the custom headers and footers
\fancyhead{} % No page header - if you want one, create it in the same way as the footers below
\fancyfoot[L]{} % Empty left footer
\fancyfoot[C]{} % Empty center footer
\fancyfoot[R]{\thepage} % Page numbering for right footer
\renewcommand{\headrulewidth}{0pt} % Remove header underlines
\renewcommand{\footrulewidth}{0pt} % Remove footer underlines
\setlength{\headheight}{13.6pt} % Customize the height of the header
\usepackage{listings}

\pgfplotsset{compat=1.10}
\numberwithin{equation}{section} % Number equations within sections (i.e. 1.1, 1.2, 2.1, 2.2 instead of 1, 2, 3, 4)
\numberwithin{figure}{section} % Number figures within sections (i.e. 1.1, 1.2, 2.1, 2.2 instead of 1, 2, 3, 4)
\numberwithin{table}{section} % Number tables within sections (i.e. 1.1, 1.2, 2.1, 2.2 instead of 1, 2, 3, 4)

\setlength\parindent{0pt} % Removes all indentation from paragraphs - comment this line for an assignment with lots of text

%----------------------------------------------------------------------------------------
%   TITLE SECTION
%----------------------------------------------------------------------------------------

\newcommand{\horrule}[1]{\rule{\linewidth}{#1}} % Create horizontal rule command with 1 argument of height

\title{ 
    \normalfont \normalsize 
    \textsc{Politechnika Warszawska} \\ [25pt] % Your university, school and/or department name(s)
    \horrule{0.5pt} \\[0.4cm] % Thin top horizontal rule
    \huge Bazy danych\\ % The assignment title
    \horrule{2pt} \\[0.5cm] % Thick bottom horizontal rule
}

\author{Mateusz Starzycki} % Your name

\date{\normalsize\today} % Today's date or a custom date

\begin{document}

\maketitle % Print the title

%----------------------------------------------------------------------------------------
%   PROBLEM 1
%----------------------------------------------------------------------------------------

\newpage
\section{Temat projektu}

Tematem wykonanego przeze mnie projektu jest system dystrybuowania zadań w bazie danych Postgres.


\newpage
\section {Wstęp}


Celem projektu było wykonanie schematu bazy danych który umożliwiałby dodawanie do kolejki zadań,
które automatyczny byłyby przekazywane do komputerów / urządzeń podrzędnych.
Komputery te są identyfikowane w bazie przez adres IP, sam mechanizm wykonania zadania nie został
zaimplementowany, zamist tego oczekiwane jest dodawanie samodzielnie rezultatów zadań.

\section {Typy danych}

W celu śledzenia czasu życia zadania, oraz stanu komputerów liczących (dalej zwanych "pracownikami"),
dodane zostały następujące dwa nowe typy danych wyliczeniowych:


\begin{lstlisting}[
           language=SQL,
           showspaces=false,
           basicstyle=\ttfamily,
           numbers=left,
           numberstyle=\tiny,
           commentstyle=\color{gray}
        ]
CREATE TYPE job_status AS ENUM (
    'new',
    'scheduled',
    'running',
    'error',
    'done'
);
\end{lstlisting}

Status ten opisuje cykl życia zadania,
\begin{itemize}
  \item new - zadanie jest nowe w systemie
  \item scheduled - zadanie trafilo do kolejki
  \item running - zadanie jest wykonywane przez pracownika
  \item error - wystąpił bład podczas wykonywania zadania
  \item done - zadanie zostało wykonane
\end{itemize}

\begin{lstlisting}[
           language=SQL,
           showspaces=false,
           basicstyle=\ttfamily,
           numbers=left,
           numberstyle=\tiny,
           commentstyle=\color{gray}
        ]
CREATE TYPE worker_status AS ENUM (
    'busy',
    'idle',
    'error',
    'not_responding'
);
\end{lstlisting}

Status ten opisuje status pracownika.

\begin{itemize}
  \item busy - pracownik jest zajęty wykonywaniem zadania
  \item idle - pracownik jest wolny do podjęcia pracy
  \item error - pracownik w stanie błędu
  \item not responding - nie można się połączyć z pracownikiem
\end{itemize}



\newpage
\section {Tabele}

Baza danych składa się z trzech tabel, zostaną one kolejno opisane w tym rozdziale.

Pierwsza z nich była to tablica zawierająca definicję zadania do wykonania o następującej definicji;

\begin{lstlisting}[
           language=SQL,
           showspaces=false,
           basicstyle=\ttfamily,
           numbers=left,
           numberstyle=\tiny,
           commentstyle=\color{gray}
        ]
CREATE TABLE jobs (
    id integer NOT NULL,
    description text NOT NULL,
    status job_status NOT NULL,
    param integer
);
\end{lstlisting}

Id było to pole klucza tabeli, opis stanowił opis zadania, status jego cykl życia oraz param pole
parametru wejściowego do zadania.


Kolejną tablicą była tablica kolejki o następującej definicji:

\begin{lstlisting}[
           language=SQL,
           showspaces=false,
           basicstyle=\ttfamily,
           numbers=left,
           numberstyle=\tiny,
           commentstyle=\color{gray}
        ]
CREATE TABLE queue (
    id integer NOT NULL,
    job integer NOT NULL
);
\end{lstlisting}

Łączyła ona zadania oraz kolejność ich wykonania, nie była możliwe priorytetowanie zadań, co można bardzo
prosto uwzględnić dodając dodatkowe polę i zmieniając funkcje przydzielającą zadania do komputerów wykonywujących.

Ostatnią tablicą była tablica resulatatów o następującej strukturze:

\begin{lstlisting}[
           language=SQL,
           showspaces=false,
           basicstyle=\ttfamily,
           numbers=left,
           numberstyle=\tiny,
           commentstyle=\color{gray}
        ]
CREATE TABLE results (
    id integer NOT NULL,
    job integer NOT NULL,
    worker integer NOT NULL,
    result integer
);
\end{lstlisting}

Tabela ta przechowuje wyniki wykonanych zadań oraz informację który pracownik wykonał zadanie oraz jego rezultat.


\newpage
\section {Klucze zewnętrzne}

Schemat posiadał następujące klucze zewnętrzne.

\begin{lstlisting}[
           language=SQL,
           showspaces=false,
           basicstyle=\ttfamily,
           numbers=left,
           numberstyle=\tiny,
           commentstyle=\color{gray}
        ]
ALTER TABLE ONLY queue
    ADD CONSTRAINT queue_job_fkey FOREIGN KEY (job)
    REFERENCES jobs(id);

ALTER TABLE ONLY results
    ADD CONSTRAINT results_job_fk_fkey FOREIGN KEY (job) 
    REFERENCES jobs(id);

ALTER TABLE ONLY results
    ADD CONSTRAINT results_woreker_fkey FOREIGN KEY (worker) 
    REFERENCES workers(id);

\end{lstlisting}

Pierwszy służył do powiązania przechowywanej w kolejce id zadania oraz jego insatncją w tabeli zdań.

Drugi łączył id wykonanego zadania z jego wpisem w tablicy zadań.

Ostatni łączył id pracownika który wykonał wynik oraz jego reprezentację w tablicy pracowników.

\newpage
\section {Procedury}

W bazie utworzone zostały następujące procedury ( wszystkie są podłączone jako triggery do tablic ):

\begin{lstlisting}[
           language=SQL,
           showspaces=false,
           basicstyle=\ttfamily,
           numbers=left,
           numberstyle=\tiny,
           commentstyle=\color{gray}
        ]
CREATE FUNCTION add_job() RETURNS trigger
    LANGUAGE plpgsql
    AS $$
    BEGIN
        IF NEW.status = 'new' THEN
          INSERT INTO queue (job) VALUES (NEW.id);
          UPDATE jobs SET status = 'scheduled' WHERE jobs.id = NEW.id;
          RETURN NEW;
        END IF;
        RETURN NEW;
    END
$$;

\end{lstlisting}

Procedura ta upewnia się iż przy dodaniu nowego zadania zostanie ono wpisane do kolejki zadań oczekujących.
Dodatkowo status zadania zostaję przez nią zmieniony z new na scheduled.



Kolejną procedurą jest procedura uruchamiana przy dodawaniu wyników przez komputery,
krok ten należy wykonać manualnie symulując długie zależności czasowe pomiędzy początkiem a końcem
wykonania zadania (należy samemu upewnić się iż wpisujemy właściwe wartości zadania oraz workera).

\begin{lstlisting}[
           language=SQL,
           showspaces=false,
           basicstyle=\ttfamily,
           numbers=left,
           numberstyle=\tiny,
           commentstyle=\color{gray}
        ]
CREATE FUNCTION add_results() RETURNS trigger
    LANGUAGE plpgsql
    AS $$
BEGIN
  UPDATE jobs SET status = 'done' WHERE jobs.id = NEW.job;
  UPDATE workers SET status = 'idle' WHERE id = NEW.worker;
  DELETE FROM queue WHERE queue.job = NEW.job;
  RETURN NULL;
END
$$;

\end{lstlisting}

Ostatnią funkcją jest funkcja przydzielająca zadanie do komputera.
Próbuję ona znaleść jaknajwięcej niezajętych pracowników i przypisać
im zadania w kolejności trafiania do kolejki.

\begin{lstlisting}[
           language=SQL,
           showspaces=false,
           basicstyle=\ttfamily,
           numbers=left,
           numberstyle=\tiny,
           commentstyle=\color{gray}
        ]
CREATE FUNCTION schedule() RETURNS trigger
    LANGUAGE plpgsql
    AS $$
DECLARE j record;
BEGIN
  FOR j IN SELECT id FROM workers WHERE status = 'idle' 
  LIMIT (SELECT count(*) FROM full_queue WHERE status = 'scheduled') 
    LOOP
      
      UPDATE jobs SET status = 'running' WHERE id = 
        (SELECT job FROM queue INNER JOIN jobs ON queue.job = jobs.id 
        WHERE jobs.status = 'scheduled' LIMIT 1);
      UPDATE workers SET status = 'busy' WHERE id = j.id;
    END LOOP;
  RETURN NULL;
  
END
$$;
\end{lstlisting}

\newpage
\section {WIdoki i zapytania}


W celu ułatwienia znajdowania problemów w procedurach utworzony został następujący widok w bazie danych:


\begin{lstlisting}[
           language=SQL,
           showspaces=false,
           basicstyle=\ttfamily,
           numbers=left,
           numberstyle=\tiny,
           commentstyle=\color{gray}
        ]
CREATE VIEW full_queue AS
 SELECT queue.id AS q_id,
    jobs.id,
    jobs.description,
    jobs.status,
    jobs.param
   FROM (queue
     LEFT JOIN jobs ON ((queue.job = jobs.id)));
\end{lstlisting}

Zapytanie to prezentuje dane zadania oraz jego pozycję w kolejcę.

\end{document}
