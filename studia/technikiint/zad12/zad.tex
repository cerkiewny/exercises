%%%%%%%%%%%%%%%%%%%%%%%%%%%%%%%%%%%%%%%%%
% Short Sectioned Assignment
% LaTeX Template
% Version 1.0 (5/5/12)
%
% This template has been downloaded from:
% http://www.LaTeXTemplates.com
%
% Original author:
% Frits Wenneker (http://www.howtotex.com)
%
% License:
% CC BY-NC-SA 3.0 (http://creativecommons.org/licenses/by-nc-sa/3.0/)
%
%%%%%%%%%%%%%%%%%%%%%%%%%%%%%%%%%%%%%%%%%

%----------------------------------------------------------------------------------------
%   PACKAGES AND OTHER DOCUMENT CONFIGURATIONS
%----------------------------------------------------------------------------------------

\documentclass[paper=a4, fontsize=11pt]{scrartcl} % A4 paper and 11pt font size

\usepackage[T1]{fontenc} % Use 10-bit encoding that has 256 glyphs
\usepackage{polski}
\usepackage[utf8]{inputenc}
\usepackage[polish]{babel}
\usepackage{amsmath,amsfonts,amsthm} % Math packages
\usepackage{pgfplots} % Math packages
\usepackage{lipsum} % Used for inserting dummy 'Lorem ipsum' text into the template
\usepackage{enumerate}
\usepackage{sectsty} % Allows customizing section commands
\usepackage{fancyhdr} % Custom headers and footers
\pagestyle{fancyplain} % Makes all pages in the document conform to the custom headers and footers
\fancyhead{} % No page header - if you want one, create it in the same way as the footers below
\fancyfoot[L]{} % Empty left footer
\fancyfoot[C]{} % Empty center footer
\fancyfoot[R]{\thepage} % Page numbering for right footer
\renewcommand{\headrulewidth}{0pt} % Remove header underlines
\renewcommand{\footrulewidth}{0pt} % Remove footer underlines
\setlength{\headheight}{13.6pt} % Customize the height of the header
\usepackage{listings}

\pgfplotsset{compat=1.10}
\numberwithin{equation}{section} % Number equations within sections (i.e. 1.1, 1.2, 2.1, 2.2 instead of 1, 2, 3, 4)
\numberwithin{figure}{section} % Number figures within sections (i.e. 1.1, 1.2, 2.1, 2.2 instead of 1, 2, 3, 4)
\numberwithin{table}{section} % Number tables within sections (i.e. 1.1, 1.2, 2.1, 2.2 instead of 1, 2, 3, 4)

\setlength\parindent{0pt} % Removes all indentation from paragraphs - comment this line for an assignment with lots of text

%----------------------------------------------------------------------------------------
%   TITLE SECTION
%----------------------------------------------------------------------------------------

\newcommand{\horrule}[1]{\rule{\linewidth}{#1}} % Create horizontal rule command with 1 argument of height

\title{ 
    \normalfont \normalsize 
    \textsc{Politechnika Warszawska} \\ [25pt] % Your university, school and/or department name(s)
    \horrule{0.5pt} \\[0.4cm] % Thin top horizontal rule
    \huge Techniki Internetu w biznesie internetowym\\ % The assignment title
    \horrule{2pt} \\[0.5cm] % Thick bottom horizontal rule
}

\author{Mateusz Starzycki} % Your name

\date{\normalsize\today} % Today's date or a custom date

\begin{document}

\maketitle % Print the title

%----------------------------------------------------------------------------------------
%   PROBLEM 1
%----------------------------------------------------------------------------------------

\newpage

\section{Zadanie 1.1}

\subsection {Pytanie}

Co to jest e-commerce i czym różni się od e-biznesu?

\subsection  {Odpowiedź}

E-commerce jest to termin określający handel internetowy.
Za handel internetwy uznawany jest hażdy biznes prowadzony w oparciu o
wspomaganie sprzedaży, dystrybucji lub kustomizacji produktu przez internet.

Pojęcie e-biznesu jest zasadniczo pojęciem szerszym, określa ono jakąkolwiek działalność
zyskową prowadzoną w oparciu lub przy pomocy internetu. E-commerce jest zatem podzbiorem
e-biznesu w którym celem działalności jest handel produktem.

\section{Zadanie 1.2}

\subsection {Pytanie}

Przedstaw własną listę wad i zalet e-biznesu.

\subsection {Odpowiedź}

Za największe zalety e-biznesu uważam:

\begin{description}
  \item[Łatwość dostępu] Klienci mogą bardzo łatwo dotrzeć do producenta.  
  \item[Globalizacja] Dotarcie do klientów z całego świata.
  \item[Dotarcie do nisz] Dzięki prowadzeniu internetowej działalności dotrzeć można do bardzo wąskich nisz które nie byłyby dostateczne do utrzymania lokalnych biznesów.
  \item[Nowe rynki] Rynki takie jak handel grami internetowymi, aukcje internetowe, analiza big data zostały utworzone dzięki rozwojowi internetu.
  \item[Łatwość reklamy] Dzięki popularności serwisów informacyjnych, wyszukiwarek, blogów specjalistycznych łatwo jest dotrzeć do klienta.
  \item[Rzetelność] Negatywna opinia biznesu bardzo szybko dociera do klientów co pozwala na zaufanie do sklepów.
\end{description}

Za największe wady e-biznesu uważam:

\begin{description}
  \item[Brak bezpośredniości] Ciężko jest dotrzeć do klienta bezpośrednio, klient zawsze ma wrażenie że rozmawia z anonimową instytucją.
  \item[Brak dostępu do produktu] Klient kupuje produkt bez uprzedniej wiedzy o jego dokładnym wyglądzie co może prowadzić do rozczarowań (w przypadku e-commarce).
  \item[Agresywny marketing] Łatwość prowadzenia zautomatyzowanego marketingu który może denerwować klientów.
  \item[Aspekty bezpieczeństwa] Internetowa przestępczość jest dużo łatwiejsza niż fizyczna, zaatakowanie jednogo serwisu może spowodować ujawnienie danych wielu klientów, zbiorowa odpowiedzialność
    firm.
  \item[Słaba legislacja] Przepisy prawne nie są dostosowane do działalności internetowych, ciężko jest wymyślić uniwersalne przepisy które nie krzywdziłyby częsci biznesów.
  \item[Rzetelność] Mimo że jest zaletą fakt iż długo działające firmy mają wypracowaną opinie klientów, nowo powstałe biznesy mogą być wyłudzeniami w które łatwo wpaść.
\end{description}


\section{Zadanie 1.3}

\subsection {Pytanie}

Z jakich elementów składa się system e-biznesu?

\subsection  {Odpowiedź}

Elementy e-biznesu to:

\begin {itemize}
  \item Handel elektroniczny
  \item analityka biznesowa
  \item zarządzanie relacjami z klientem
  \item zarządzanie łańcuchem dostaw
  \item planowanie zasobów przedsiębiorstwa
\end {itemize}

\subsubsection {Handel elektroniczny} 

Działalność oparta na sprzedaży produktu poprzez internet.

\subsubsection {Analityka biznesowa}

Działalność oparta na zbieraniu. zarządzaniu oraz preszukiwaniu informacjami w celu wspomagania decyzji biznesowych.

\subsubsection {Zarządzanie relacjami z klientem}

Działalonść polegająca na budowaniu trwałej relacji z klientem. Próba utworzenia trwałej więzi z klientem polegającej na zrozumieniu jego potrzeb oraz wspomaganie klientów (lub przyszłych
klientów w rozwiązywaniu ich.

\subsubsection {Zarządzanie łańcuchem dostaw}

Realizacja szybkich elementów obsugi procesu wykonania produktu w celu szybkiej reakcji na zmianę rynku, minimalizacje kosztów produkcji i innych aspektów związanych z ulepszeniem procesu produkcji.

\subsubsection {Planowanie zasobów przedsiębiorstwa}

Część odpowiedzialna za automatyzacje zarządzani zasobami przedsiębiorstwa, takimi jak finansowe oraz materialne.

\section{Zadanie 1.4}

\subsection {Pytanie}

Jakie znasz podstawowe cechy przedsięwzięć e-biznesowych w zakresie relacji z klientem?

\subsection  {Odpowiedź}
\begin {itemize}
   \item wirtualizacja produktu,
   \item indywidualizacja kompozycji wartości,
   \item usieciowienie,
   \item marketing doświadczeń,
   \item wykorzystanie cyklu życia klienta.
\end {itemize}

\subsubsection {Wirtualizacja produktu}

Wirtualizacja produktu polega na przeniesienie go do strefy cyfrowej. Czasami nie jest możliwe całkowite przeniesienie produktu do cyberprzestrzeni dlatego innym podejściem wirtualizacji produktu
jest wzbogacenie go o informacje, dobrym przykładem może być odzież sportowa, która wzbogacona jest uwcześnie o serwisy nadzorujące oraz podsumowujące treningi.

\subsubsection {indywidualizacja kompozycji wartości} 

Indywidualizacja kompozycji polega na dostosowaniu produktu dokładnie do potrzeb klienta. Dobrym przykładem może być firma apple oferująca dowolnie wygrawerowany wzór na produkcie iPad,
w przypadku klienta kupującego urządzenie jako prezent dla konkretnej osoby jest to indywidualizacja kompozycji - może w skrajnym przypadku nie istnieć drugi taki sam produkt.

\subsubsection {usieciowienie}

Usieciowienie polega na odniesieniu wartości produktu poprzez interakcję z innymi jego posiadaczami.
Najlepszym przykładem są gry online które utworzyły osobny rynek konsumencki.

\subsubsection {Marketing doświadczeń}

Marketing doświadczeń polega na dostarczeniu doświadczeń stymulujących pozytywne emocje u klienta. Jest to metoda prowadzenia działalności konkurencyjnej do innych firm nie tylko poartej na jakość 
dostarczanego produktu. 

\section{Zadanie 1.5}

\subsection {Pytanie}

W jaki sposób Twoim zdaniem Internet może stymulować przedsiębiorczość?

\subsection  {Odpowiedź}

Internet pomaga zwiększyć opłacalność niektórych działalnościm które z powodu małego procentowo zainteresowania nie byłyby opłacalne, jednakża możliwość dotarcia globalnego do klientów pozwala im
na rozwój. Kolejnym przykładem pozytywnej stymulacji przedsiębiorczości jest zwiększenie informacji oraz konkurencyjności usługodawców. Równy dostęp do informacji pozwala na wybranie dostosowanych
do potrzeb działalności dostawców usług. Innym aspektem stymulacji jest zmniejszenie kosztów potrzebnych na administrację firmą oraz klientami.
Dodatkowym aspektem jest otwrcie nowych rynków konsumenckich opartych o analize danych itd.
Ostatnim aspektem jest wspomniany w wykładzie propaganda konkurencyjności która przenika do każdej gałęzi przemysłu oraz dostępność informacji które wpływają na powstawanie nowych pomysłow na 
działalność

\section{Zadanie 2.1}

\subsection {Pytanie}

Który z modeli biznesu internetowego wg Twojej opinii, powinien być wiodący dla polskich MŚP i dlaczego?

\subsection  {Odpowiedź}

Moim zdaniem, w Polsce, jako kraju rozwijającym się najważniejszymi modelami działalności internetowej są:

\begin{itemize}
  \item Kreatorzy rynku
  \item Sklepy i przetargi elektroniczne
  \item Handel elektroniczny B2C
\end{itemize}

Z tych wymienionych, biorąc pod uwagę fakt iż Polska jest krajem o dużym rozwarstwieniu społecznym najwżniejszym modelem wg mnie
powinna byc kreacja nowych rynków. Daje to możliwość zaistnienia nie tylko w kraju ale także na świecie. Działalności oparte o sprzedaż internetową
są siłą rzeczy oparte o firmy zgraniczne produkujące różne przedmioty, nie daje to zatem pełnej możliwości rozwoju dla kraju.

\section{Zadanie 2.2}

\subsection {Pytanie}

Zakładając, że Twoja firma produkuje bombki choinkowe i zamierza je eksportować, który z modeli biznesowych byś wykorzystał?

\subsection  {Odpowiedź}

Biorąc pod uwagę taką działalność skupiłbym się na modelu biznesowym handel elektroniczny B2B. Ewentualnie handel elektroniczny B2C.
Dużą zaletą handlu B2B jest zmniejszony koszt dystrybucji produktu - po znalezieniu lokalnych biznesów nie trzeba przesyłać produktu
do indywidualnych klientów. Pomijając fakt iż łatwo zawrzeć wtedy umowę z firmami transportowymi, handel B2B zapewnia dużo większą
ciągłość rynku, optymalizuje koszty, ułatawia zarządzanie firmą.

\section{Zadanie 2.3}

\subsection {Pytanie}

Jakie główne grupy czynników wpływają na przychody gospodarki internetowej?

\subsection  {Odpowiedź}

Główne grupy czynników wpływające na przychody gospodarki internetowej to czynniki zewnętrzne oraz wewnętrzne.

Do grupy czynników zewnętrznych zaliczamy:

\begin {itemize}
  \item Liczbę użytkowników
  \item Poziom konwersji użytkownika
  \item Średni przychód jednostkowy
\end {itemize}

Do grupy czynników wewnętrznych zaś:

\begin {itemize}
  \item Wielkość rynku 
  \item Konkurencję
  \item Produkty i usługi substytucyjne
  \item Szacowaną skłonność do zakupu przez grupę docelową 
\end {itemize}

\section{Zadanie 2.4}

\subsection {Pytanie}

Które czynniki Twoim zdaniem mają najważniejszy wpływ na funkcjonowanie firm internetowych w Polsce?

\subsection  {Odpowiedź}

Moim zdaniem czynnikami wpływającymi w największej mieże na funkcjonowanie firm internetowych jest niski średni przychód jednostkowy,
oraz wielkość rynku. W związku z tym iż jesteśmy średniej wielkości krajem o niskim przychodzie jednostkowym (jako kraj rozwijający się),
bardzo istotnym czynnikiem jest wybranie takiej grupy docelowej aby maksymalizować wielkość rynku. Dodatkowym czynnikiem jest fakt iż w Polsce
społeczeństwo bardzo szybko starzeje się co wpływa także na liczbę użytkowników używających internetu w codziennym życiu.

\section{Zadanie 2.5}

\subsection {Pytanie}

Które rozwiązania biznesowe w sieci są Twoim zdaniem najbardziej korzystne? Uzasadnij swoją odpowiedź.

\subsection  {Odpowiedź}

Moim zdaniem najbardziej korzystnym biznesem internetowym są elektroniczne centrum handlowe pozwalają one na bardzo łatwo skalowalny biznes.
Gwarantują (powyżej pewnego progu) sukces na dużą skalę - oferują wszystkim trzem stroną benefity - klientom biznesu: poszczególne działalności są
bardzo łatwo dostępne z poziomu centra handlowego, gwarantują też dużą konkurencje; firmą członkowskim: praktycznie darmowa reklama z tytułu przynależenia
do danego centrum, zaufanie klienta; firmy "rodzica": zarobek zależny zwykle od obrotu (im popularniejsze staje się centrum tym bardziej wszyscy chcą z niego korzystać,
zwiększa to możliwości przy negocjacji umów z firmami oraz prestiż. Dobrym przykładem jest firma Valve która z firmy robiącej gry stała się dostawcą najbardzej
znanej platformy do zakupu gier: Steam. 

Jedyną wadą tego rodzaju działalności jest to iż gdy się nie wiedzie prymu w danym sektorze bardzo ciężko przeskoczyć lidera.

\end{document}
