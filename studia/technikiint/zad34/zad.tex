%%%%%%%%%%%%%%%%%%%%%%%%%%%%%%%%%%%%%%%%%
% Short Sectioned Assignment
% LaTeX Template
% Version 1.0 (5/5/12)
%
% This template has been downloaded from:
% http://www.LaTeXTemplates.com
%
% Original author:
% Frits Wenneker (http://www.howtotex.com)
%
% License:
% CC BY-NC-SA 3.0 (http://creativecommons.org/licenses/by-nc-sa/3.0/)
%
%%%%%%%%%%%%%%%%%%%%%%%%%%%%%%%%%%%%%%%%%

%----------------------------------------------------------------------------------------
%   PACKAGES AND OTHER DOCUMENT CONFIGURATIONS
%----------------------------------------------------------------------------------------

\documentclass[paper=a4, fontsize=11pt]{scrartcl} % A4 paper and 11pt font size

\usepackage[T1]{fontenc} % Use 10-bit encoding that has 256 glyphs
\usepackage{polski}
\usepackage[utf8]{inputenc}
\usepackage[polish]{babel}
\usepackage{amsmath,amsfonts,amsthm} % Math packages
\usepackage{pgfplots} % Math packages
\usepackage{lipsum} % Used for inserting dummy 'Lorem ipsum' text into the template
\usepackage{enumerate}
\usepackage{sectsty} % Allows customizing section commands
\usepackage{fancyhdr} % Custom headers and footers
\pagestyle{fancyplain} % Makes all pages in the document conform to the custom headers and footers
\fancyhead{} % No page header - if you want one, create it in the same way as the footers below
\fancyfoot[L]{} % Empty left footer
\fancyfoot[C]{} % Empty center footer
\fancyfoot[R]{\thepage} % Page numbering for right footer
\renewcommand{\headrulewidth}{0pt} % Remove header underlines
\renewcommand{\footrulewidth}{0pt} % Remove footer underlines
\setlength{\headheight}{13.6pt} % Customize the height of the header
\usepackage{listings}

\pgfplotsset{compat=1.10}
\numberwithin{equation}{section} % Number equations within sections (i.e. 1.1, 1.2, 2.1, 2.2 instead of 1, 2, 3, 4)
\numberwithin{figure}{section} % Number figures within sections (i.e. 1.1, 1.2, 2.1, 2.2 instead of 1, 2, 3, 4)
\numberwithin{table}{section} % Number tables within sections (i.e. 1.1, 1.2, 2.1, 2.2 instead of 1, 2, 3, 4)

\setlength\parindent{0pt} % Removes all indentation from paragraphs - comment this line for an assignment with lots of text
\renewcommand\thesection{}
\renewcommand\thesubsection{}

%----------------------------------------------------------------------------------------
%   TITLE SECTION
%----------------------------------------------------------------------------------------

\newcommand{\horrule}[1]{\rule{\linewidth}{#1}} % Create horizontal rule command with 1 argument of height

\title{ 
    \normalfont \normalsize 
    \textsc{Politechnika Warszawska} \\ [25pt] % Your university, school and/or department name(s)
    \horrule{0.5pt} \\[0.4cm] % Thin top horizontal rule
    \huge Techniki Internetu w biznesie internetowym\\ % The assignment title
    \horrule{2pt} \\[0.5cm] % Thick bottom horizontal rule
}

\author{Mateusz Starzycki} % Your name

\date{\normalsize\today} % Today's date or a custom date

\begin{document}

\maketitle % Print the title

%----------------------------------------------------------------------------------------
%   PROBLEM 1
%----------------------------------------------------------------------------------------

\newpage

\section{Zadanie 3.1}

\subsection {Pytanie}

Wymień i krótko scharakteryzuj elementy procesu orgranizacji kampanii reklamowych on -line.

\subsection {Odpowiedź}

Etapami kampani reklamowych są:

\begin {itemize}
  \item planowanie
  \item przygotowanie
  \item realizacja
  \item ocena skuteczności
\end {itemize}

Planowanie kampanii polega na wybraniu grupy docelowej, technologii narzędzi marketingowych oraz strategii.
Dodatkowym bardzo ważnym czynnikiem jest ustalenie celów kampanii. W tym etapie przygotowuje się także budżet
oraz kosztorys przedsięwzięcia.

Przygotowanie kampanii polega na przygotowaniu docelowej formy, określeniu w jaki sposób będzie zrealizowana
reklama, gdzie zostanie umieszczona w jaki sposób i w jakich serwisach będzie ona emitowana itd.

Realizacja kampanii polega na wdrożeniu planu.

Ocena skuteczności jest etapem podsumowującym mającym na celu wyciągnięcie wniosków na przyszłość.
Podsumowane zostają dobre oraz złe cechy organizacji oraz przebiegu kampanii, dodatkowo oceniana jes zyskowność
oraz długotrwały wpływ kampanii na wizerunek firmy.

\section{Zadanie 3.2}

\subsection {Pytanie}

Którą z form internetowej reklamy graficznej uważasz za najlepszą. Odpowiedź uzasadnij.

\subsection {Odpowiedź}

Za najlepszą formę reklamy graficznej uważam formę skyscrapper, jest ona dość ładnie wkomponowana w wąskie zwykle
strony internetowe, dodatkowo treść jej nie jest aż tak rażąca i łatwo na nią nie zwrócić uwagi, co za tym idzie
jest najmniej inwazyjną formą prezentacji graficznej produktu w internecie, jako taka sprawia że dużo większe jest
prawdopodobieństwo iż przy przeczytaniu jej ktoś na prawdę jest zainteresowany produktem - łatwo zobaczyć każdy baner
na stronie, ale gdy są one zbyt widoczne i zakłócają treść, wzbudzają one tylko niechęć internauty.

\section{Zadanie 3.3}

\subsection {Pytanie}

Czym różni się brand marekting od performance marketingu?

\subsection {Odpowiedź}

Brand marketing jest to forma marketingu oparta o budowę marki - głównym celem jest zbudowanie długotrwałego wizerunku
firmy. Bardzo ważne jest dobre imię oraz zbudowanie takiej wizji firmy która najbardziej zachęci grupę docelową klientów.

Performance marketing jest oparty na szybkie zwiększenie sprzedaży lub zyskowności firmy. Głównym celem perfomrance marketingu
jest zatem rozszerzenie kręgu sprzedaży, rozbudowanie rynku, zachęcenie do kupowania produktu jego ceną lub innymi walorami.
Nie ma w takiej formie miejsca na budowanie długotrwałych wartości związanych z firmą.

\section{Zadanie 3.4}

\subsection {Pytanie}

Przedstaw własną listę wad i zalet e-marketingu?

\subsection {Odpowiedź}

Ogromną zaletą e-marketingu jest bardzo łatwe zwrócenie uwagi setek tysięcy a nawet milionów użytkowników swoim produktem.
Jest to według mnie przekaz który ma największe szanse dotarcia do dużej ilości ludzi. 

Kolejną zaletą jest możliwość kierowania marketingu jedynie do potencjalnie zainteresowanych klientów, możliwe jest profilowanie
użytkowników na podstawie ich zachowań w sieci.

Wadą jest bardzo duża agresywność reklam i złe mniemanie użytkników, każdy z nas był w sytuacji gdy ma swieżo otwartą stronę internetową
gdy nagle wyskakuje mu reklama zabierająca cały ekran, lub w której obejrzenie 10 sekundowego filmiku poprzedzone jest 40 sekundami obowiązkowymi
do obejrzenia reklamami.

Drugą wadą jest nieetyczność zabiegów stosowanych przez niektóre firmy. Agresywny marketing który jest nietrafiony jest bardzo irytujący oraz
może zniszczyć dobre imię firmy.

\section{Zadanie 3.5}

\subsection {Pytanie}

Znajdź w sieci ankietę i prześledź jej budowę. Jakie są jej Elementy? Czu odpowiada
ona budowei tradycyjnego kwestionariusza?

\subsection {Odpowiedź}

Ankieta którą analizuję znajduje się pod adresem: http://www.ankietka.pl/ankieta/7036/ankieta-dotyczaca-internetu.html.
Jest to ankieta dotycząca użycia internetu. Składa się ona z trzech części, pierwszą z nich jest część wprowadzająca
znajdują się w niej pytania dotyczące grupy wiekowej oraz społecznej, w pierwszej części zgrubnie klasyfikuję swoją przynależność
poprzez odpowiedź na zgrubne pytania o wiek płeć wykształcenie. Jest to część która pozwala na wyciągnięcie wniosków, zasadniczo
bez niej badanie które ma na celu określic charakter użytkownika sieci byłoby bez sensu.

Druga część dotyczy podstawowych zagadnień związanych z tematem, pytania dotyczą tego w jakich ilościach i w jaki sposób korzystam
z internetu, z jakich narzędzi korzystam. Część ta zarówno jak i poprzednia dotyczą faktów.

Ostatnia część to określenie moich poglądów na pewne zagadnienia, pytania te dotyczą już nie faktów tyle co szczegółowej opinii.
Moim zdaniem ankieta ta może pokazać wiele cennych informacji, między innymi jak opinie na przyszłość komunikacji postrzegana
jest na przekroju różnych warstw społecznych, wieku oraz na tle różnej edukacji.

\section{Zadanie 3.6}

\subsection {Pytanie}

Jakie są Twomim zdaniem zalety i wady reklamy tradycyjnej i reklamy internetowej?

\subsection {Odpowiedź}

Moim zdaniem zaletami reklamy internetowej jest dotarcie do bardzo szerokiego grona odbiorców za przystępną cenę.
Kolejną zaletą jest możliwość targetowania reklam (o czym wspomniałem przy okazji e marketingu).
Dodatkową zaletą jest jej różnorodność pod względem formy.
Jej efektywność przy odpowiednim stosowaniu formy oraz dobranie strategii może być bardzo duża, może to być zaletą,
jest to według mnie również i wadą, bardzo ciężko trafić w gusta odbiorców i możliwe jest wywołanie dużego zniechęcenia
do firmy lub produktu.
Dodatkową wadą jest duże zniechęcenie użytkowników do takiej formy. Wiąże się to z przesyceniem niektórych serwisów
reklamami.

\section{Zadanie 3.7}

\subsection {Pytanie}

Skonstruuj sondaż lub ankietę internetową na temat sposobu spędzania wolnego czasu przez
studentów. Zaplanuj badanie. Jakie są argumenty za, a jakie przeciw przeprowadzaniu tych badań
on-line.

\subsection {Odpowiedź}

Ankieta podzielona byłaby na następujące częsci:

Pytania charakteryzujące osobę:
\begin{enumerate}
  \item Wiek
  \item Rok studiów 
  \item Uczelnia 
  \item Kierunek studiów 
  \item Płeć 
  \item Czy poza studiamy masz pracę 
\end{enumerate}

Pytania dotyczące ogólnego wykorzystania czasu wolnego:
\begin{enumerate}
  \item Jak dużo czasu wolnego masz w tygodniu
  \item Jak dużo czasu wolnego masz w weekendy
  \item Jak często uprawiasz sport
  \item Jak często wyjeżdżasz w celach rekreacyjnych na cały weekend
  \item Jak często wykorzystujesz wolny czas na realizację hobby
  \item W jaki dodatkowy czas spędzasz swój czas wolny (pytanie otwarte)
\end{enumerate}

Pytania dotyczące użytkowania komputera:
\begin{enumerate}
  \item Jak często używasz internetu
  \item Jak dużo czasu spędzasz na graniu w gry
  \item Jak dużo czasu spędzasz na portalach społecznych
  \item Jak dużo czasu spędzasz na graniu online
\end{enumerate}

Pytania dotyczące otoczenia:
\begin{enumerate}
  \item Jak często twój wolny czas jest spędzony z rodziną
  \item Jak często twój wolny czas spędzony jest z przyjaciółmi
  \item Jakie są aktywności czasu wolnego które najczęściej wykonujesz z przyjaciółmi (wymień 3 i podaj ile godzin w tygodniu średnio trwają)
\end{enumerate}


Argumentami za są:
\begin{enumerate}
  \item Łatwa dostępność
  \item Różnorodność użytkowników
  \item Cena przeprowadzenia badania
\end{enumerate}

Wadami zaś

\begin{enumerate}
  \item Wybiórczość (większe prawdopodobieństwo na wypełnienie przez ludzi spełniających wolny czas w internecie)
  \item Mała rzetelność - ciężko wymusić uczciwe wypełnienie przez użytkowników
  \item Ograniczenia długości - łatwo zniechęcić użytkownika do wypełniania
\end{enumerate}

\section{Zadanie 3.8}

\subsection {Pytanie}

Wymień i opisz mechanizm budowania marki w serwisie WW i poza nim.

\subsection {Odpowiedź}

Mechanizm budowania marki w serwisie WW oraz poza nim różni się zasadniczo od siebie.
W działalności klasycznej marka budowana jest poprzez np przywiązanie do pewnej grupy reprezentującej markę.
Reklamy w różnej formie nagrywane są przez tych samych aktorów, ich wizerunek staje się twarzą firmy.
Dodatkowym aspektem jest przywiązanie wartości do firmy klasyczny przykład to np precyzja, przywiązanie do
działalności, długoletnia tradycja itd. Dodatkowo w klasycznym marketingu stosuje się przywiązanie do
znaku firmowego, loga, cytatu podsumowującego firmę, przykładem może być "Zawsze coca-cola" czy 
"Biedronka, codziennie niskie ceny". 

Budowanie wizerunku firmy w serwisie WW polega natomiast na rozszerzeniu informacji o firmie (np serwisy informacyjne
na temat filmów, ciekawostki). Dodatkowym przykładem jest rozszerzanie świata firmy aby stał się przeżyciem oraz
doświadczeniem dla każdego "wtajemniczonego" w życie produktu ( np serial "Archer" umieścił szereg zagadek w filmie które
kontynuowały się przez liczne serwisy WWW po to aby znaleść stronę jednego z bohaterów serialu). Dodatkowym przykładem jest
wzbudzenie pozytywnych emocjii, np reklama "Old spice" stała się niejako symbolem samym w sobie, firma dostarczyła
dodatkowo serwis www w którym użytkownik mógł grać na mięśniach bohatera abstrakcyjnej reklamy.

\section{Zadanie 4.1}

\subsection {Pytanie}

Z jakich elementów składa się typowy e-sklep?

\subsection {Odpowiedź}

Typowy e-sklep składa się z dwóch głównych części.

Pierwszą jest back-end w którego skład wchodzą:

\begin{itemize}
  \item Administracja użytkownikami
  \item Administracja treścią 
  \item Administracja produktami 
  \item Administracja kategoriami 
  \item Obsługa sprzedaży 
\end{itemize}

Drugą jest zaś front-end; strona obsługująca klientów
w jej skład wchodzi:

\begin{itemize}
  \item Strona główna
  \item Strona kategorii produktu
  \item Strona poszczególnych produktów
  \item Strona koszyka
\end{itemize}

\section{Zadanie 4.2}

\subsection {Pytanie}

Jakie znasz modele sklepów internetowych?

\subsection {Odpowiedź}

Modele sklepów internetowych to:
\begin{itemize}
  \item Samodzielne sklepy www
  \item Działalności w ramach pasażu handlowego
  \item Forma mieszna
  \item Hipermarkety internetowe
\end{itemize}

\section{Zadanie 4.3}

\subsection {Pytanie}

Wymień i opisz trzy pakiety oprogramowania obsługujące sklepy internetowe.

\subsection {Odpowiedź}

Znanymi mi systemami obsługującymi sklepy internetowe są:
\begin{itemize}
  \item Joomla eshop
  \item Django shop module
  \item Paypal
\end{itemize}

Joomla eshop jest to bardzo łatwy system zarządzania treścią, jest to darmowy
open sourcowy produkt. Dodatek do niego pozwala na dodawanie produktów z ceną 
oraz pogrupowanych w kategorie do strony. Dodatkowo moduł umożliwia także śledzenie
historii zamówień, danych użytkownika, wsparcie sprzedaży oraz logistyki.

Django shop module jest bardzo podobnym modułem opartym o technologie Django w języku python.
Django jest to także darmowy troche bardziej zaawansowany framework tworzenia stron internetowych
zapewnia on dużo bardziej nisko poziomowy dostęp do kontroli przepływu informacji na stronie.

Paypal jest to serwis obsługujący płatności online pobiera on 3\% każdej transkacji oraz umożliwia bardzo
łatwą autentykacje użytkowników. Zaletami serwisu jest łatwość jego obsługi oraz bardzo łatwa intergacja
z kartami płatnościowymi oraz jego szeroka rozpoznawalność.


\section{Zadanie 4.4}

\subsection {Pytanie}

Wymień podobieństwa i różnice pomiędzy bankiem tradycyjnym a wirtualnym.

\subsection {Odpowiedź}

Największym podobieństwem banków tradycyjnych oraz wirtualnych jest ich model finansowania
oparty na płatności przy przelewie (procent od transakcji walutowej lub stała opłata).
Kolejnym podobieństwem jest koncepcja posiadania konta oraz środków.

Główną różnicą jest ilość pieniędzy trzymanych na kontach w bankach tradycyjnych a wirtualnych.
Wirtualne serwisy bankowe zwykle posiadają kwoty znacznie mniejsze. Dodatkową różnicą jest 
brak możliwość pobierania kredytów w bankach internetowych. Najważniejszą według mnie różnicą
jest według mnie fakt iż wirtualne banki jeszcze nie doprowadziły do bankructwa żadnego kraju.

\section{Zadanie 4.5}

\subsection {Pytanie}

Wymień rodzaje aukcji internetowych. Która Twoim zdaniem jest najlepsza z punktu widzenia
kupującego, a któraz z punktu widzenia sprzedającego? Odpowiedź uzasadnij.

\subsection {Odpowiedź}

Aukcje internetowe dzielimy na aukcje:
\begin{itemize}
  \item Klasyczne
  \item Holenderskie
  \item Natychmiastowe
  \item Jednokrotne i wielokrotne
  \item Błyskawiczne
  \item Przetargowe
  \item Odwrócone
  \item Wertykalne i horyzontalne
\end{itemize}

Moim zdaniem najlepszym typem aukcji dla kupującego jest aukcja klasyczna. Przy ustaleniu ceny
minimalnej ma on pewność iż nie straci na licytowanym przedmiocie, wszystko natomiast ponad kwotę
ustaloną jako minimum jest jego zyskiem, co przy dużym popycie może osiągnąć znacznie więcej niż
cena minimalna.

Najlepszą aukcją dla kupującego jest natomiast aukcja odwrócona, znajduje się on zasadniczo w sytuacji
sprzedającego w aukcji klasycznej. W obu przypadkach często zdaża się iż ludzie płacą za produkt więcej
niż zapłaciliby w przypadku gdy targowaliby się ze sprzedawcą, bardzo łatwo jest bowiem podwyżać cenę
maksymalną o pare złotych powyżej swojego maksimum  w przypadku gdy zależy nam na przedmiocie.


\section{Zadanie 4.6}

\subsection {Pytanie}

W dowolnych trzech sklepach prześledź regulamin zakupów. Czy znalazły się tam informację o prawach
klienta (możliwość zwrotu lub wymiany towarów itp)? Znajdź informację o kontroli polskich e-sklepów.

\subsection {Odpowiedź}

Prześledzone przeze mnie regulaminy sklepów to regulaminy serwisu : Allegro, empik, wolczanka.

Wszystkie te serwisy opisują prawa klienta, opisują informację na temat zwrotu produktu zarówno
nieuszkodzonego jak i wadliwego.

Informację o prawach konsumenta znajdują się w ustawie z dnia 30 maja 2014 roku o prawach przysługujących konsumentowi.

\section{Zadanie 4.7}

\subsection {Pytanie}

Znajdź w sieci oferty sklepów oferujących żywność. Przeanalizuj ich liczbę ofertę i zasięg.

\subsection {Odpowiedź}

Sklepów oferujących żywność są setki tysięcy.

Ich zasięg oraz oferta różni się zasadniczo. Można podzielić je na trzy grupy.

Pierwszą z nich są usługi lokalne. Mają one bardzo ograniczoną ofertę np kebaby lokalne
grille, restauracje. Ich zasięg wacha się od paru do parunastu kilometrów.

Kolejną grupą są biznesy regionalne, mają one bardziej zróżnicowaną oferte i zasięg parudziesięciu
kilometrów, są to np pizzerie regionalne.

Ostatnią grupą są supermarkety spożywcze oraz sieciowe restauracje, mają one zasięg globalny w 
skali kraju, oferta ich różni się zakresem dla serwisów takich jak KFC jest mocno ograniczona
dla supermarketów jest znacznie bardziej zróżnicowana

\section{Zadanie 4.8}

\subsection {Pytanie}

Porównaj i oceń zawartość dwóch małych przedsiębiorstw produkujących meble.

\subsection {Odpowiedź}

Porównywane serwisy to http://www.meble-z-drewna.pl oraz http://www.agatameble.pl.

Pierwszy serwis oferuje specyficznie wykonane meble z litego drewna. Drugi serwis oferuje
dużo tańsze meble wykonane z materiałów drewno podobnych lub oklejonej dykty. 
Zasadniczą różnicą pomiędzy serwisami, pomijając dobrane barwy oraz wygląd, jest ich zakres oferty
oraz grupa docelowa. Pierwszy serwis oferuje dużo węższą gamę produktów o wysokiej jakości, co za tym
idzie za grupę docelową obiera średnią klasę społeczną. Drugi serwis jest dużo bardziej bogaty w ofercie,
produkowane meble są znacznie tańsze, przez co w celu opłacalności produkcji zmuszeni są do większej
ilościowo sprzedaży.

Oba serwisy uważam a bardzo estetyczne oraz w przystępny sposób przedstawiające zawartość sklepu.
Uważam za lepszy serwis meble-z-drewna, oferta sklepu wydaj mi się dużo bardziej atrakcyjna, dodatkowo
dobrane przez nich wygląd strony bardziej przemawia do mojego poczucia estetyki.


\section{Zadanie 4.9}

\subsection {Pytanie}

Znajdź w w sieci oferty dwóch firm prowadzących taką samą działalność, z podobnym zakresem terytorialnym.
Porównaj i wysuń wnioski do jakich klientów skierowana jest ich oferta.

\subsection {Odpowiedź}

Rozpatrywane serwisy to: http://www.fabrykamarzen.waw.pl/ oraz http://lasvegasstyle.pl/.

Oba serwisy prezentują ofertę fryzjerską na terenie Warszawy. Pierwszy z nich już od samego wejścia do serwisu
sprawia wrażenie ekskluzywnego serwisu, prezentuje wizję artystyczną skupia się na opisaniu ekipy pracującej w serwisie
zakresu usługi oraz opisu referencji medialncyh

Drugi serwis prezentuje portfolio swoich osiągnięć jako zdjęcia z salonu fryzjerskiego, jedna z zakładek od razu
opisuje ceny usług, serwis skupia się na prezentacji wnętrza salonu oraz opinii klientów.

Pierwszy serwis ma według mnie trafić do ludzi zainteresowanych modą, osób bogatych, prawdopodobnie publicznych
lub związanych z mediami, dla których cena usługi nie jest tak ważna jak jej jakośc.

Drugi serwis natomiast celuje według mnie w ludzi młodych próbujących dobrać fryzurę na weekendową imprezę w 
przystępnej cenie.

\section {4.10}

\subsection {Pytanie}

Znajdź stronę małej firmy o lokalnym zasięgu, średniej firmy i międzynarodowej korporacji. Czy Twoim zdaniem
sieć jest narzędziem dużym, średnim, czy małym przedsiębiorstwom? W jakim zakresie każde z tych przedsiębiorstw
może i powinno wykorzystywać internet?

\subsection {Odpowiedź}

Moim zdaniem internet jest uniwersalnym narzędziem wykorzystywanym przez firmy różnej wielkości.
Każda z firm, w zależności od wielkości wykorzystuje go jako narzędzie do osiągnięcia innych celów.
Małe i średnie przedsiębiorstwa generalnie wykorzystują internet jako dodatkowy sposób sprzedaży.
Dodatkowo ułatwia to często administracją firmy. Duże przedsiębiorstwa lub międzynarodowe korporacje
bardzo często wykorzystują internet jako źródło reklamy, dodatkowo w związku z globalnym charakterem
sieci WW firmy prowadzące działaność opartą o internet mają tendencję do szybkiego rozrastania się.

Bardzo częstym błędem popełnianym przez małe przedsiębiorstwa jest wykorzystywanie internetu jako źródła
reklamy. Z powodu małych finansów niestety reklamy takie są bardzo często nieskuteczne, dodatkowo
brak wiedzy i ustawienia dobrego odbiorcy reklamy powoduje bardzo mało skuteczną reklame, wydaje mi się
że korporacje oraz średnie przedsiębiorstwa dość dobrze wykorzystują internet.

\section{Zadanie 4.11}

\subsection {Pytanie}

Załóżmy że masz zamiar kupić aparat fotograficzny i wodoszczelną obudowę do robienia zdjęc pod wodą.
Znajdź potrzebne infromację, w tym charakterystykę funkcji, opinie użytkowników, opisy modeli oraz ceny.
Przeanalizuj proces wyboru produktu w porównaniu z procesem realizowanym w tradycyjny sposób.

\subsection {Odpowiedź}

Proces dobierania aparatu przy użyciu internetu dużo bardziej przypomina według mnie proces nauki.
Przy początku wyboru kierowałem się jedynie ceną zestawu przy porównywalnych parametrach. Nie wiedziałem jednak
które z nich są istotne, miałem niejakie wyobrażenie o tym iż istotną cechą jest szczelność obudowy i rozdzielczość
apratu. Podczas czytania opinie o aparatach o podobnych parametrach dowiedziałem się natomiast że głównym problemem
robieniem zdjęc pod wodą jest światło, przy dużej rozdzielczości niektórych aparatów nie będe w stanie wykonać zdjęć
na większej głębokości gdyż przesłona aparatu uniemożliwi mi na dostanie się odpowiedniej ilości światła do matrycy.
Po przejrzeniu opinii użytkowników dowiedziałem się bardzo wielu istotnych rzeczy których firm należy unikać.

W przypadku klasycznego wybierania bardzo problematyczne jest zebranie tak kompletnej informacji na temat produktu.
Zwykle kontaktuje się jedynie ze sprzedawcą jednego sklepu który porówna jedynie aparaty które ma w ofercie, dodatkowo
jeśli nie mam odpowiedniej widzy na temat przedmiotu nie poświęci on wystarczającej ilości czasu na skupienie się na
szczegółach które mnie mogą nie interesować skupi się jedynie na rzeczach które uważam za istotne aby doprowadzić do
kupna produktu.

\section{Zadanie 4.12}

\subsection {Pytanie}

Scharakteryzuj polskie firmy obecne w sieci. W jakim celu firmy te wykorzystują internet?

\subsection {Odpowiedź}

Według firmy Google, najpopularniejszymi firmami prowadzącymi serwisy internetowe w polsce są serwisy informacyjne
oraz serwis aukcyjny allegro. Są to firmy najpopularniejsze w polskim rynki, z wyjątkiem serwisu allegro są to praktycznie
tylko serwisy informacyjne oraz rozrywkowe.

Ranking: http://interaktywnie.com/biznes/artykuly/raporty-i-badania/100-najpopularniejszych-stron-w-polsce-wedlug-google-nasza-klasa-poza-czolowka-18365.

\section{Zadanie 4.13}

\subsection {Pytanie}

Jakie grupy produktów są najpowszechniej oferowane w sieci? Jak sądzisz dlaczego?

\subsection {Odpowiedź}

Grupą najczęściej oferowanych serwisó w sieci są gry komputerowe oraz sprzęt elektronczny.
Sądze iż są dwa powody, po pierwsze profil w który się wpisuje ze względu na spędzanie 
dużej ilości czasu grając z przyjaciółmi powoduje iż taki właśnie typ reklam jest mi przedstawiany.
Dodatkowym aspektem jest także wstępny filtr - aby zobaczyć te reklamy trzeba mieć internet w związku
z czym trzeba mieć sprzęt elektroniczny, ma się też predyspozycje w postaci komputera do grania w gry.
Ostatnim aspektem jest wielkość rynku gamingowego który ostanio zdecydowanie wzrasta.

\section{Zadanie 4.14}

\subsection {Pytanie}

Znajdź w internecie najnowsze  wyniki badań dotyczących znajomości portali i serwisów internetowych. Jak można wykorzystać te informacje w przedsiębiorstwie?

\subsection {Odpowiedź}

\section{Zadanie 4.15}

Badania dotyczące popularności portali i serwisów znajdują się na http://www.infomonitor.pl/download/e-commerce-w-polsce-2014.pdf.
Informację tę można wykorzystać na wiele sposobów. Można na przykład targetować podobnych klientów jak najpopularniejsze serwisy,
dodatkowo można próbować umieścić w takim serwisie reklamę firmy. Innym sposobem może być udoskonalenie serwisu poprzez obserwację
i analizę sukcesu innych witryn.

\subsection {Pytanie}

Jednym z mitów na temat Internetu jest przekonanie, że sieć umożliwia odniesienie sukcesu każdej firmie, także zupełnie nieznanej lub dopiero rozpoczynającej działalność.
Jak sądzisz, czy to prawda czy fałsz? Odpowiedź uzasadnij.

\subsection {Odpowiedź}

Uważam iż mit ten jest mocno przesadzony. Wywodzi się on z sukcesów takich firm jak Facebook czy Google.
Istnieje dużo przypadków w których internet może zdecydowanie przyspieszyć rozwój przedsiębiorstwa.
Nie jest to jednak uniwersalna forma rozwoju działalności. Sam obecnie pracuje w małym start upie.
Zajmujemy się produkcją sprzętu prezentującego przy pomocy ultradźwięków dotyku na ręce użytkownika.
Internet jest dla nas zdecydowanie nienajlepszą formą reklamy głównie ze względu na model biznesowy.
Głównymi klientami naszej firmy są ogromne koncerny, a co za tym idzie dotarcie do nich poprzez marketing
internetowy jest praktycznie niemożliwe. W przypadku innej działąlności, prowadzonej przez moją bliską znajomą
reklama w internecie bardzo przyspieszyła rozwój jej przedsięwzięcia. Jej firma zajmuje się produkcją biżuterii, 
w ciągu miesiąca od wypozycjonowaniu w wyszukiwarce jej firma przynisła dziesięciokrotnie większe zyski.
Podsymowując wydaje mi się iż mit ten jest mocno przesadzony z powodu firm odnoszących spektakularne sukcesy
dzięki marketingowi, jest to także mechanizm samonapędzający się - ludzie dużo chętnie przekazują sobie informację
na temat sukcesy firmy niż jej updaku. Dodatkowo firma która odniosła sukces dzięki marketingowi internetowemu
jest dobrze znana wszystkim, gdyż na tym polega jej sukces.

\end{document}
