%%%%%%%%%%%%%%%%%%%%%%%%%%%%%%%%%%%%%%%%%
% Short Sectioned Assignment
% LaTeX Template
% Version 1.0 (5/5/12)
%
% This template has been downloaded from:
% http://www.LaTeXTemplates.com
%
% Original author:
% Frits Wenneker (http://www.howtotex.com)
%
% License:
% CC BY-NC-SA 3.0 (http://creativecommons.org/licenses/by-nc-sa/3.0/)
%
%%%%%%%%%%%%%%%%%%%%%%%%%%%%%%%%%%%%%%%%%

%----------------------------------------------------------------------------------------
%   PACKAGES AND OTHER DOCUMENT CONFIGURATIONS
%----------------------------------------------------------------------------------------

\documentclass[paper=a4, fontsize=11pt]{scrartcl} % A4 paper and 11pt font size

\usepackage[T1]{fontenc} % Use 10-bit encoding that has 256 glyphs
\usepackage{polski}
\usepackage[utf8]{inputenc}
\usepackage[polish]{babel}
\usepackage{amsmath,amsfonts,amsthm} % Math packages
\usepackage{pgfplots} % Math packages
\usepackage{lipsum} % Used for inserting dummy 'Lorem ipsum' text into the template
\usepackage{enumerate}
\usepackage{sectsty} % Allows customizing section commands
\usepackage{gensymb}
\usepackage{fancyhdr} % Custom headers and footers
\pagestyle{fancyplain} % Makes all pages in the document conform to the custom headers and footers
\fancyhead{} % No page header - if you want one, create it in the same way as the footers below
\fancyfoot[L]{} % Empty left footer
\fancyfoot[C]{} % Empty center footer
\fancyfoot[R]{\thepage} % Page numbering for right footer
\renewcommand{\headrulewidth}{0pt} % Remove header underlines
\renewcommand{\footrulewidth}{0pt} % Remove footer underlines
\setlength{\headheight}{13.6pt} % Customize the height of the header
\usepackage{listings}

\pgfplotsset{compat=1.10}
\numberwithin{equation}{section} % Number equations within sections (i.e. 1.1, 1.2, 2.1, 2.2 instead of 1, 2, 3, 4)
\numberwithin{figure}{section} % Number figures within sections (i.e. 1.1, 1.2, 2.1, 2.2 instead of 1, 2, 3, 4)
\numberwithin{table}{section} % Number tables within sections (i.e. 1.1, 1.2, 2.1, 2.2 instead of 1, 2, 3, 4)

\setlength\parindent{0pt} % Removes all indentation from paragraphs - comment this line for an assignment with lots of text

%----------------------------------------------------------------------------------------
%   TITLE SECTION
%----------------------------------------------------------------------------------------

\newcommand{\horrule}[1]{\rule{\linewidth}{#1}} % Create horizontal rule command with 1 argument of height

\title{ 
    \normalfont \normalsize 
    \textsc{Politechnika Warszawska} \\ [25pt] % Your university, school and/or department name(s)
    \horrule{0.5pt} \\[0.4cm] % Thin top horizontal rule
    \huge Projekt z przedmiotu fizyczne podstawy transmisji i przechowywania informacji \\ % The assignment title
    \horrule{2pt} \\[0.5cm] % Thick bottom horizontal rule
}

\author{Mateusz Starzycki} % Your name

\date{\normalsize\today} % Today's date or a custom date

\begin{document}

\maketitle % Print the title
\newpage
\tableofcontents
\newpage
%----------------------------------------------------------------------------------------
%   PROBLEM 1
%----------------------------------------------------------------------------------------

\newpage


\section{Treść Zadania}

Wiedząc, że moc nadajnika wynosi \(P_N\), sprawność fotodiody wynosi \(X\), zaprojketuj jak najdłuższe łącze światłowodowe jednokanałowe
(działające w trzecim oknie telekomunikacyjnym) o przepływności B (nie używając wzmacniaczy). Załóż że straty na spawach i połączeniach wynoszą odpowiednio
\(A_s\) - 1 spaw i \(A_c\) - 1 złącze. Wybierz trzy różne odległości między spawami z przedziału [5;10] km. Omów wpływ odległości między spawami na zasięg łącza.
Potrzebne parametry włókna światłowodowego znajdź na stronie producenta. Omów poszczególne etapy projektu i wpływ parametrów łącza na jego zasięg.

\section{Teoria transmisji światłowodowej}

\subsection{Wstęp}

Transmisja światłowodowa jest to przesył informacji za pomocą wiązki elektromagnetycznej z zakresu fal widzialnych.
Sygnał przesyłany jest przy użycia nosnika który utrzymuje w sobie falę przy wykorzystaniu całkowitego wewnętrznego odbicia.
Nośnik taki nazywamy światłowodem.
Aby możliwe było powstanie tego zjawiska swiatłowód musi zostać wykonany z przynajmniej dwóch ośrodków o róznej prędkości propagacji światła.

\subsection{Zjawiska występujące w Światłowodzie}

W światłowodzie zachodzi wiele zjawisk fizycznych, można je podzielić na zjawiska porządane i negatywnie wpływające na transmisję sygnału.

\subsection{Całkowite wewnętrzne odbicie}

Najważniejszym, porządanym zjawiskiem jest opisane we wstępnie zjawisko całkowitego wewnętrznego odbicia. Polega ono na odbiciu promienia świetlnego od
krawędzi dwóch ośrodków bez możliwości przeniknięcia do drugiego ośrodka. Aby zjawisko to miało miejsce muszą zostać spełnione dwa warunki:
\begin{itemize}
  \item Ośrodek w którym światło się znajduje musi mieć większy współczynnik załamania niż ośrodek od którego odbija się światło.
  \item Kąt padania musi być większy od pewnego kąta krytycznego.
\end{itemize}

Kąt krytyczny jest to taki kąt w którym wiązka odbijająca się w granicy dwóch ośrodków musiałaby po odbiciu przebiegać wzdłóż granicy ośrodków.
Można zwizualizować to na następującym schemacie:

Z prawa załamania można obliczyć zatem kąt odbicia.\[n_1sin\Theta_1=n_2sin\Theta_2\] 
Podkładając \[\Theta_{gran}=90\degree\] 
Otrzymujemy: \[n_1sin\Theta_C=n_2sin\Theta_{gran}\]
Po przekształceniu możemy wyliczyć kąt graniczny z wzoru \[\Theta_C=arcsin\frac{n_2}{n_1}\]

Zjawisko to jest wykorzystywane w światłowodzie gdyż sprawia iż sygnał jest całkowicie "uwięziony" w wiązce, pod warunkiem że wpada do niego
z odpowiednim kątem i pod warunkiem że światłowód nie jest mocno zgięty w żadnym fragmencie.

\subsection{Tłumienie}

W związku z tym iż nośnikiem informacji są fotony o określonej długości fali, na drodze światła dochodzi do jego rozproszenia przez jądra atomowe oraz absorpcji poprzez
elektrony. Zjawisko to powoduje osłabienie sygnału w zależności od przebytej drogi w światłowodzie. 

Intensywność absorpcji fotonów zależy od długości fali i jest ustalone przez rodzaj wykorzystanego źródła sygnału w światłowodzie. Następujący rysunek prezentuje
zależność absorpcji (uwzględniający oba zjawiska) od długości fali. Na rysunku zostały naniesione linie (od lewej) intensywności rozproszen Rayleight'a oraz absorpcji podczerwieni.

\subsection{Straty na złączach i spawach}

Dodatkowymi czynnikami tłumienia są niedoskonałości wykonania światłowodu. Zjawiska takie występują miedzy innymi na złaczach końcowych oraz spawach pomiędzy
fragmentami światłowodów, ma to związek z lokalnymi zaburzeniami geometrii które są dużo większe niż w innych miejscach światłowodu.

\subsection{Dyspersja}

Dodatkowym zjawiskiem występującym w światłowodzie jest dyspersja. Zjawisko to występuje z powodu różnicy dróg przebytej przez poszczególne fale.
W związku z tym iż odbijają się one wielokrotnie od krawędzi światłowodu początkowo prostokątny sygnał intensywności w funkcji czasu zostaje rozmyty.
Jego ostre krawędzie zostają rozmyte w czasie. Rysunek przedstawia zjawisko dyspersji, wykres intensywności sygnału w czasie w oddalających się od źródła
fragmentach światłowodu.




\end{document}

