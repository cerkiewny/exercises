%%%%%%%%%%%%%%%%%%%%%%%%%%%%%%%%%%%%%%%%%
% Short Sectioned Assignment
% LaTeX Template
% Version 1.0 (5/5/12)
%
% This template has been downloaded from:
% http://www.LaTeXTemplates.com
%
% Original author:
% Frits Wenneker (http://www.howtotex.com)
%
% License:
% CC BY-NC-SA 3.0 (http://creativecommons.org/licenses/by-nc-sa/3.0/)
%
%%%%%%%%%%%%%%%%%%%%%%%%%%%%%%%%%%%%%%%%%

%----------------------------------------------------------------------------------------
%   PACKAGES AND OTHER DOCUMENT CONFIGURATIONS
%----------------------------------------------------------------------------------------

\documentclass[paper=a4, fontsize=11pt]{scrartcl} % A4 paper and 11pt font size

\usepackage[T1]{fontenc} % Use 10-bit encoding that has 256 glyphs
\usepackage[english]{babel}
\usepackage{polski}
\usepackage[utf8]{inputenc}
\usepackage{amsmath,amsfonts,amsthm} % Math packages
\usepackage{pgfplots} % Math packages
\usepackage{lipsum} % Used for inserting dummy 'Lorem ipsum' text into the template
\usepackage{enumerate}
\usepackage{sectsty} % Allows customizing section commands
\allsectionsfont{\centering \normalfont\scshape} % Make all sections centered, the default font and small caps
\usepackage{fancyhdr} % Custom headers and footers
\pagestyle{fancyplain} % Makes all pages in the document conform to the custom headers and footers
\fancyhead{} % No page header - if you want one, create it in the same way as the footers below
\fancyfoot[L]{} % Empty left footer
\fancyfoot[C]{} % Empty center footer
\fancyfoot[R]{\thepage} % Page numbering for right footer
\renewcommand{\headrulewidth}{0pt} % Remove header underlines
\renewcommand{\footrulewidth}{0pt} % Remove footer underlines
\setlength{\headheight}{13.6pt} % Customize the height of the header

\numberwithin{equation}{section} % Number equations within sections (i.e. 1.1, 1.2, 2.1, 2.2 instead of 1, 2, 3, 4)
\numberwithin{figure}{section} % Number figures within sections (i.e. 1.1, 1.2, 2.1, 2.2 instead of 1, 2, 3, 4)
\numberwithin{table}{section} % Number tables within sections (i.e. 1.1, 1.2, 2.1, 2.2 instead of 1, 2, 3, 4)

\setlength\parindent{0pt} % Removes all indentation from paragraphs - comment this line for an assignment with lots of text

%----------------------------------------------------------------------------------------
%   TITLE SECTION
%----------------------------------------------------------------------------------------

\newcommand{\horrule}[1]{\rule{\linewidth}{#1}} % Create horizontal rule command with 1 argument of height

\title{ 
    \normalfont \normalsize 
    \textsc{Politechnika Warszawska} \\ [25pt] % Your university, school and/or department name(s)
    \horrule{0.5pt} \\[0.4cm] % Thin top horizontal rule
    \huge Metody Optymalizacji zadanie 1\\ % The assignment title
    \horrule{2pt} \\[0.5cm] % Thick bottom horizontal rule
}

\author{Mateusz Starzycki} % Your name

\date{\normalsize\today} % Today's date or a custom date

\begin{document}

\maketitle % Print the title

%----------------------------------------------------------------------------------------
%   PROBLEM 1
%----------------------------------------------------------------------------------------

\section{Zadanie 1.1}

Dane:

Do wyprodukowania 1 jednostki towaru I potrzeba 50 jednostek surowca i 2 roboczogodziny a towaru 2 20 j surowca i 5 roboczogodzin.
Ograniczenia surowców: 25000 jednostek surowca i 2000 roboczogodzin.
Cena jednostki surowca wynosi 20 zł roboczogodziny 100zł cena 1 i 2 jednostki towaru wynosi 1500zł.

Szykane:

Maksymalny zysk.

Równania:

\[x_1 * 1500 - 20 * 50 * x_1 - 2 * 100 * x_1 + x_2 * 1500  - 20 * 20 * x_2 - 5 * 100 * x_2 = Z\]

Po uproszczeniu:

\[x_1 * 1200 + 600 * x_2 = Z\]

Zadanie optymalnego wyboru:

\[Z -> max\]

Ograniczenia:

\[x_1 >= 0\]
\[x_2 >= 0\]
\[s_1 + s_2 <= 25000 \]
\[r_1 + r_2 <= 2000\]
\[s_1 = 50 * x_1\]
\[s_2 = 20 * x_2\]
\[r_1 = 2 * x_1\]
\[r_2 = 5 * x_2\]

Gdzie:
\begin{itemize}
\item $ x_1 $ to ilość produktu 1 
\item $ x_2 $ to ilość produktu 2
\item $ r_1 $ to ilość roboczogodzin zużytych podczas produkcji produktu 1 
\item $ r_2 $ to ilość roboczogodzin zużytych podczas produkcji produktu 2
\item $ s_1 $ to ilość surowca zużytego podczas produkcji produktu 1 
\item $ s_2 $ to ilość surowca zużytego podczas produkcji produktu 2
\item [Z] to Zysk
\end{itemize}

\begin{tikzpicture}
\begin{axis}
[
    title={Contour plot, view from top},
    view={0}{90}
]
\addplot3[
    contour gnuplot={levels={0.8, 0.4, 0.2, -0.2}}
]
{sin(deg(sqrt(x^2+y^2)))/sqrt(x^2+y^2)};
\end{axis}
\end{tikzpicture}
%------------------------------------------------

\subsection{Heading on level 2 (subsection)}

Lorem ipsum dolor sit amet, consectetuer adipiscing elit. 
\begin{align}
A = 
\begin{bmatrix}
A_{11} & A_{21} \\
A_{21} & A_{22}
\end{bmatrix}
\end{align}
Aenean commodo ligula eget dolor. Aenean massa. Cum sociis natoque penatibus et magnis dis parturient montes, nascetur ridiculus mus. Donec quam felis, ultricies nec, pellentesque eu, pretium quis, sem.

%------------------------------------------------

\subsubsection{Heading on level 3 (subsubsection)}

\lipsum[3] % Dummy text

\paragraph{Heading on level 4 (paragraph)}

\lipsum[6] % Dummy text

