%%%%%%%%%%%%%%%%%%%%%%%%%%%%%%%%%%%%%%%%%
% Short Sectioned Assignment
% LaTeX Template
% Version 1.0 (5/5/12)
%
% This template has been downloaded from:
% http://www.LaTeXTemplates.com
%
% Original author:
% Frits Wenneker (http://www.howtotex.com)
%
% License:
% CC BY-NC-SA 3.0 (http://creativecommons.org/licenses/by-nc-sa/3.0/)
%
%%%%%%%%%%%%%%%%%%%%%%%%%%%%%%%%%%%%%%%%%

%----------------------------------------------------------------------------------------
%   PACKAGES AND OTHER DOCUMENT CONFIGURATIONS
%----------------------------------------------------------------------------------------
\documentclass[paper=a4, fontsize=11pt]{scrartcl} % A4 paper and 11pt font size

\usepackage[polish]{babel}
\usepackage[utf8]{inputenc}
\usepackage[T1]{fontenc}
\DeclareUnicodeCharacter{00A0}{ }
\DeclareUnicodeCharacter{00A0}{~}
\usepackage{lmodern}
\selectlanguage{polish}
\usepackage{amsmath,amsfonts,amsthm} % Math packages
\usepackage{enumerate}
\usepackage{graphicx}
\DeclareGraphicsExtensions{.pdf,.png,.jpg}
\usepackage{enumitem}
\setlength\parindent{0pt} % Removes all indentation from paragraphs - comment this line for an assignment with lots of text
\usepackage{tabularx}
\usepackage{pgfplots}
\usepackage{listings}

\newcommand{\horrule}[1]{\rule{\linewidth}{#1}}

\title{ 
    \normalfont \normalsize 
    \textsc{Politechnika Warszawska} \\ [25pt] % Your university, school and/or department name(s)
    \horrule{0.5pt} \\[0.4cm] % Thin top horizontal rule
    \huge Metody Optymalizacji zestaw zadań 2\\ % The assignment title
    \horrule{2pt} \\[0.5cm] % Thick bottom horizontal rule
}

\author{Mateusz Starzycki} % Your name

\date{\normalsize\today} % Today's date or a custom date

\begin{document}

\maketitle % Print the title

%----------------------------------------------------------------------------------------
%   PROBLEM 1
%----------------------------------------------------------------------------------------

\newpage
\section{Zadanie 2.1}

Tabela przedstawiająca wartości pomiarów:
\newline  
\newline  
{
  \begin{tabular}{|c|c|c|}
    \hline
    numer pomiaru &  x & y \\
    \hline
    1 & -3 & 1 \\
    \hline
    2 & 1 & 3 \\
    \hline
    3 & 2 & 5 \\
    \hline

  \end{tabular}
}
\newline  
\newline  
Celem zadania jest znalezienie najlepszego dopasowania prostej do pomiarów.
Metodą wskazaną jest użycie metody najmniejszych kwadratów

\[y=\alpha x + \beta\]

Z wzorów w metodzie najmniejszych kwadratów możemy wyliczyć odpowiednio:
\[\alpha = \frac{SS_{xy}-S_x*S_y}{\delta}\]
Oraz:
\[\beta = \frac{S_{xx}S_y-S_xS_{xy}}{\delta}\]
Gdzie:
\[\delta=S*S_{xx}-S_x^2\]
\[S=n\]
\[S_x=\sum\limits_{i=1}^nx_i=0\]
\[S_y=\sum\limits_{i=1}^ny_i9\]
\[S_{xx}=\sum\limits_{i=1}^nx^2_i=14\]
\[S_{xy}=\sum\limits_{i=1}^nx_iy_i=10\]
\[S_{yy}=\sum\limits_{i=1}^ny^2_i=35\]


Podkładając te wartości do wzorów otrzymujemy odpowiednio:

\[\alpha = 0.71\]
\[\beta = 3\]

Wykres minimalizowanej funkcji to:
Oś pozioma przedstawia wartość \[\alpha\] natomiast pionowa wartość \[\beta\].
\begin{tikzpicture}
  \begin{axis}[
  title={$Funkcja minimalizowana$},
  domain=0:4,
  view={0}{90},
  ]
  \addplot3[contour gnuplot={number=24},thick]
  {(1-(-3*x+y))^2+(3-(x+y))^2+(5-(2*x+y))^2};
  \end{axis}
\end{tikzpicture}

Wzór na funkcję minimalizowaną:

\[(1-(-3*x+y))^2+(3-(x+y))^2+(5-(2*x+y))^2\]

\newpage
\section{Zadanie 2.2}

Funkcja badana:

\[f(x,y)=20x+26y+4xy-4x^2-3^y2\]

Poszukujemy maksimum tej funkcji, albo minimum funkcji danej wzorem:

\[f'(x,y)=-20x-26y-4xy+4x^2+3^y2\]

Ograniczenia to x,y > 0.

Badając pochodne cząstkowe funkcji:

\[\frac{\partial f'}{\partial x}=-20-4y+8x\]
\[\frac{\partial f'}{\partial y}=-26-4x+6y\]

Otrzymujemy potencjalny punkt na minimum: \[(\frac{56}{9},\frac{67}{9})\].

Wartość funkcji w tym punkcie wynosi 182.1.

Pozostaje udowodnienie że funkcja ta jest wklęsła.

W tym celu sprawdzimy wartość pochodnej:
\[\frac{\partial f'^2}{\partial x^2}=8\]

Oraz
\[\frac{\partial f'^2}{\partial y^2}=6\]

Ponieważ są one obie niezależne od współrzędnych x lub y, funkcja musi być wklęsła zarówno w osi x jak i y (jej rzut na obie płaszczyzny jest parabolą). 

\newpage
\section{Zadanie 2.3}

Pisząc prosty algorytm przeszukiwania siły brutalnej:

\begin{lstlisting}
def main():
   i = 0
   j = 0

   maxj = 0
   maxim = 0
   while(i * 4000 < 200000):
        j = 0
        while((i * 4000 + j * 5000) < 200000):
            if((12*i*j - i*i -3*j*j) > maxim):
                maxim = (12*i*j - i*i -3*j*j)
                maxi = i
                maxj = j
            j = j + 1
        i= i + 1
    print ( maxim )
    print ( maxi )
    print ( maxj )
\end{lstlisting}

Otrzymujemy wartość S równą : 4169, dla ilości reklam w gazecie 26 oraz 19 minutach reklam w telewizji.
\newpage
\section{Zadanie 2.5}

Funkcja badana:

\[f(x)=\mathopen|x\mathclose|+ x^4\]

Aby zbadać wypukłość tej funkcji należy podstawić jej wartości do zbioru:

\[(\forall \alpha 0<\alpha<1) (\forall x^0, x^1 \in X) f(\alpha x^0+(1-\alpha)x^1)\leq\alpha f(x^0)+(1-\alpha)f(x^1)\]

Podstawiając funkcję:

\[\mathopen|\alpha x^0 + (1 - \alpha)x^1\mathclose|+ (\alpha x^0+(1-\alpha)x^1)^4 \leq\alpha \mathopen|x^0\mathclose|+ (x^0)^4+(1-\alpha)\mathopen|x^1\mathclose|+ (x^1)^4\]

Wiemy że:

\[\mathopen|\alpha x^0 + (1 - \alpha)x^1\mathclose| \leq (1-\alpha)\mathopen|x^1\mathclose| + \alpha\mathopen|x^0\mathclose|\]

Zostaje do wykazania:

\[(\alpha x^0+(1-\alpha)x^1)^4 \leq(x^0)^4+ (x^1)^4\]

Co łatwo można sprowadzić do postaci:

\[(\alpha x^1+(1-\alpha)x^1)^4 \leq(x^0)^4+ (x^1)^4\]

Zakładając że wartość bezwględna \[x^1\] jest większa od \[x^0\].
W tym wypadku jeden z członów skróci się pozostawiając trywialną nierówność.

W drugim przypadku posłużymy się twierdzeniem iż suma funkcji wypukłych jest funkcją wypukłą.
Obliczając drugą pochodną funkcji \[exp\] wiemy że jest ona wypukła (2 pochodna zawsze większa od 0).
To samo możemy powiedzieć o funkcji \[exp(-\frac{1}{2}x)\]) jest ona bowiem przeskalowana oraz obrócona względem osi Y.

Funkcja \[x^4\] jest także funkcją wypukła (jej druga pochodna to \[12x^2\] która jest zawsze większa od zera.
A zatem ich suma również musi być funkcją wypukłą.
\newpage
\section{Zadanie 2.6}

Dana jest funkcja:

\[x\in\mathbb{R}^2f(x)=(x_1)^2+x_1x_2+3(x_2)^2\]

Jej gradient to:

\[\Delta f(x)=[\frac{\partial f}{\partial x_1},\frac{\partial f}{\partial x_2}]=[2x_1+x_2,6x_2+x_1]\]

Pochodna w kierunku d=(1,1):

\[\Delta f(x) * d = 3x_1 + 7x_2\]

Pochodna w kierunku p=(-1,1):

\[\Delta f(x) * d = -x_1 + 5x_2\]

Macierz Hessego:
\[
\begin{bmatrix}
  \frac{\partial f}{\partial x_1^2} & \frac{\partial f}{\partial x_2x_2} \\
  \frac{\partial f}{\partial x_1x_2} & \frac{\partial f}{\partial x_2^2} 
 \end{bmatrix}
=
\begin{bmatrix}
  2 & 1 \\
  1 & 6 
 \end{bmatrix}
 \]
\newpage
\section{Zadanie 2.7}

Dana jest macierz funkcji kwadratowej:

\[
\begin{bmatrix}
  3 & 1 \\
  2 & 4 
\end{bmatrix}
\]

Pierwszym krokiem stwierdzenia czy funkcja jest wypukła jest wyliczenie równania charakterystycznego:

\[
det \begin{bmatrix}
  s - 3 & -1 \\
  -2 & s - 4 
\end{bmatrix}
= s^2 - 7s + 10 
\]

Rozwiązania to:
\[s_1=2\]
\[s_2=5\]

Jest to zatem macierz dodatnio określona, funkcja jest wypukła.
Kroki te powtarzamy dla pozostałych macierzy:

\[
det\begin{bmatrix}
  s+3 & -1 \\
  -2 & s+4 
\end{bmatrix}
= s^2 + 7s +10
\]

Rozwiązania to:
\[s_1=-2\]
\[s_2=-5\]
Jest to zatem macierz ujemnie określona, nie jest zatem wypukła.


\[
det\begin{bmatrix}
  s-6 & -2 \\
  -1 & s+2 
\end{bmatrix} = s^2 -4s - 14
\]

Rozwiązania to:
\[s_1=2+3\sqrt{2}\]
\[s_2=2-3\sqrt{2}\]

Jest to zatem macierz ujemnie nieokreślona, nie jest zatem wypukła.

\[
det\begin{bmatrix}
  s-3 & 2 \\
  0 & s-2 
\end{bmatrix}
= s^2 -5s
\]
Rozwiązania to:
\[s_1=0\]
\[s_2=5\]
Macierz ta jest zatem dodatnio półokreślona, funkcja jest wypukła
\newpage
\section{Zadanie 2.8}

Funkcja dana jest wzorem:
\[f(x)=0.25x^4-x^3+x^2+1\]

Wyliczam pierwszą oraz drugą jej pochodną:

\[f'(x)=x^3-3x^2+2x\]

\[f''(x)=3x^2-6x+2\]

Pierwsza pochodna zeruje się w punkcie 0 (x można wyciągnąć przed wszystkie składniki)


\[f'(x)=x*(x^2-3x+2)\]

Zgadując dalej pierwiastki, wielomian w nawiasie ma pierwiastek w punkcie 1.


\[f'(x)=x*(x-1)(x-2)\]

Ostatni pierwiastek musi zatem istnieć w punkcie 2.

Licząc wartość drugiej pochodnej w tych punktach:

\[f''(0)=2\]
\[f''(1)=3-6+2=-1\]
\[f''(2)=3*4-6*2+2=12-12+2\]

Zatem funkcja posiada 2 minima lokalne w punkcie 0 oraz 2. Co zresztą można zobaczyć na wizualizacji funkcji: \\

\begin{tikzpicture}
  \begin{axis}[
    title={$Wizualizacja funkcji$},
  ]
  % use TeX as calculator:
  \addplot[domain=-1:3] {0.25*x^4-x^3+x^2+1};
  \end{axis}
\end{tikzpicture}

\newpage


\section{Zadanie 2.9}


Funkcja dana jest wzorem:
\[f(x)=4x^3-18x^2+27x-7\]

Wyliczam pierwszą oraz drugą jej pochodną:

\[f'(x)=12x^2-36x+27\]

\[f''(x)=24x-36\]

Jedynym, podwójnym pierwiastkiem równania kwadratowego jest punkt \[ \frac{3}{2}\].

Oznacza to że funkcja, będąc funkcją trzeciego stopnia nie posiada minimum lokalnego a jedynie punkt przegięcia.

Można to potwierdzić również sprawdzając rysunek w okolicy punktu 3/2.

\begin{tikzpicture}
  \begin{axis}[
    title={$Wizualizacja funkcji$},
  ]
  % use TeX as calculator:
  \addplot[domain=1:2] {4*x^3-18*x^2+27*x-7};
  \end{axis}
\end{tikzpicture}

\newpage



\section{Zadanie 2.10}

Znaleść aproksymacje funkcji:

\[f(x_1,x_2) = (x_1-1)^2exp(x_2)+x_1\]

w punkcie (0,0) oraz (1,1)

Zaczniemy od policzenia pierwszej oraz drugiej pochodnej funkcji (zakładam aproksymacje kwadratową).

\[\frac{\partial f'}{\partial x_1} = 2(x_1-1)exp(x_2)\]
\[\frac{\partial f'}{\partial x_2} = (x_1-1)^2exp(x_2)\]
\[\frac{\partial f''}{\partial x_1^2} = 2exp(x_2)\]
\[\frac{\partial f''}{\partial x_1x_2} = (x_1-1)exp(x_2)\]
\[\frac{\partial f''}{\partial x_2^2} = (x_1-1)^2exp(x_2)\]
Zaczniemy od punktu (0,0)

\[f(0,0) = (0-1)^2*exp(0)+0=1\]
Znając wartość w punkcjie możemy przejść do obliczenia pochodnej w tym punkcie.
\[\frac{\partial f'}{\partial x_1} = 2(0-1)exp(0)=-2\]
\[\frac{\partial f'}{\partial x_2} = (0-1)^2exp(0)=1\]
\[\frac{\partial f''}{\partial x_1^2} = 2exp(0)=2\]
\[\frac{\partial f''}{\partial x_1x_2} = (0-1)exp(0)=-1\]
\[\frac{\partial f''}{\partial x_2^2} = (0-1)^2exp(0)=1\]

Następnie stosując szereg tailora możemy przybliżyc funkcje jako:

\[f(\Delta x_1,\Delta x_2)=1-2\Delta x_1+\Delta x_2+\Delta x_1^2-\Delta x_1\Delta x_2+\frac{1}{2}\Delta x_2^2\]



w punkcie (1,1) natomiast:

\[f(1,1) = (1 - 1)^2*exp(1)+1 = 1\]
Znając wartość w punkcjie możemy przejść do obliczenia pochodnej w tym punkcie.
\[\frac{\partial f'}{\partial x_1} = 2(1-1)exp(1)=0\]
\[\frac{\partial f'}{\partial x_2} = (1-1)^2exp(1)=0\]
\[\frac{\partial f''}{\partial x_1^2} = 2exp(1)=0\]
\[\frac{\partial f''}{\partial x_1x_2} = (1-1)exp(1)=0\]
\[\frac{\partial f''}{\partial x_2^2} = (1-1)^2exp(1)=0\]

Następnie stosując szereg tailora możemy przybliżyc funkcje jako:

\[f(\Delta x_1,\Delta x_2)=1\]

\begin{tikzpicture}
  \begin{axis}
    \addplot3[surf,domain=0:2] {(x-1)^2*exp(y)+x};
  \end{axis}
\end{tikzpicture}

\end{document}
