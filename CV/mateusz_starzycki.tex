%% start of file `template.tex'.
%% Copyright 2006-2010 Xavier Danaux (xdanaux@gmail.com).
%% Copyright 2010-2011 Mark Liu (markwayneliu@gmail.com).
%
% This work may be distributed and/or modified under the
% conditions of the LaTeX Project Public License version 1.3c,
% available at http://www.latex-project.org/lppl/.

\documentclass[8pt,a4paper]{moderncv}

\usepackage{verbatim}

% moderncv themes
\moderncvtheme[blue]{classic}                 % optional argument are 'blue' (default), 'orange', 'red', 'green', 'grey' and 'roman' (for roman fonts, instead of sans serif fonts)
%\moderncvtheme[green]{classic}                % idem

% character encoding
\usepackage[utf8]{inputenc}                   % replace by the encoding you are using

% adjust the page margins
\usepackage[scale=0.8]{geometry}
%\setlength{\hintscolumnwidth}{3cm}						% if you want to change the width of the column with the dates
%\AtBeginDocument{\setlength{\maketitlenamewidth}{6cm}}  % only for the classic theme, if you want to change the width of your name placeholder (to leave more space for your address details
%\AtBeginDocument{\recomputelengths}                     % required when changes are made to page layout lengths


% personal data
\firstname{Mateusz}
\familyname{Starzycki}
\address{586 Southmead Road}{BS10 5NH Bristol}    % optional, remove the line if not wanted
\mobile{+44 7425696040}                    % optional, remove the line if not wanted
\email{mstarzycki@gmail.com}                      % optional, remove the line if not wanted
%\extrainfo{\url{http://markliu.me}} % optional, remove the line if not wanted

% to show numerical labels in the bibliography; only useful if you make citations in your resume
%\makeatletter
%\renewcommand*{\bibliographyitemlabel}{\@biblabel{\arabic{enumiv}}}
%\makeatother

%\nopagenumbers{}                             % uncomment to suppress automatic page numbering for CVs longer than one page
%----------------------------------------------------------------------------------
%            content
%----------------------------------------------------------------------------------

\begin{document}
\maketitle

\section{Skills}
\subsection{Proficient With}
\cvline{C}{Kernel source reading. Camera Driver maintenance and development. Programming course on studies. Self studies programming under POSIX standard in Linux. Linux porting and driver development course in Samsung.}
\cvline{C++}{Study course. development using boost and QT libraries}
\cvline{Python}{Commercial Django project. University project under PyGTK for raspberry Pi board.}
\cvline{Technologies}{Driver development, Django, PostgreSQL, MySQL, \LaTeX{}, Bash Scripting, Git, Vim, Ubuntu, Fedora, POSIX, Gcc, Perforce, stdlib, Linux libraries}
\subsection{Have Experience With}
\cvline{Languages}{PHP, Java, VHDL}
\cvline{Technologies}{Apache, SQL Server, Eclipse, Raspdebian, Windows, Boost, Samba, Make, GDB, ADB, ARM Assembler}

\section{Languages}
\cvline{Polish}{Native}
\cvline{English}{Good speaking an writing skill}
\cvline{German}{Medium speaking and writing skill}

\section{Relevant Coursework}
\subsection{Warsaw University of technology}
\cvline{Computer Science}{Programming in C++, Digital Design, VHDL, CAD tools, Computer Network Fundamentals, Programming Languages, Java, Neural networks and AI algorithms, Digital signal processing, Boolean logic, Automatics, Robotics}
\cvline{Electrical Engineering}{Electromagnetism, Semiconductors, Systems and Signals, Circuit Analysis}
\cvline{Self Studies}{Operating Systems, Django, Device, Driver, Development, Complexity, theory, Algorithms}
\cvline{Internet technologies}{Routing basic, Switching, Dial-up connections}
\subsection{Samsung}
\cvline{Git}{Git usage course}
\cvline{Linux Porting}{Linux new device porting course}

\section{Education}
\vspace{4mm}
\cventry{2008--2013}{Bachelor of Engineering, Mechatronics Studies}{\newline Warsaw University of technology}{Warsaw, PL}{}{}
\cvline{Advisor:}{\small Professor Jakub Zmigrodzki}
\cvline{Thesis:}{Graphical user interface for experimental Doppler Flowmeter based on ARM micro controller}
\cvline{}{}
Mechatronical studies introduce designing and manufacturing complex devices and concern about in three science fields: mechanics, electronics and optics.
During seven semesters I learned the basic concept of electronic circuits, mechanical and micro mechanical device manufacturing and design.
My studies introduced also hardware preparation and software development on created platform.
During the studies I had chance to test my knowledge in few projects. First of them, focused on the
mechanical design introduced creation mechanical device - laboratory linear table. Next project introduced electronic design of analog circuit for voice amplifier.
Further projects introduced software development: High level - Java gui development based on AWK libraries - Navier-Stokes solver, low level programming - Holter EKG device development (team project) and 
final thesis project introduced usage of all skills mention above, it was entirely created with Python under Raspberry Pi board, using PyGTK Libraries on Raspdebian operating system.
Project included the design and manufacturing of SPI to parallel communication converter.
\cvline{}{}
\cvline{}{}
\vspace{4mm}
\cventry{2005--2008}{Experimental Mathematical Program class}{\newline Stanisław Staszic High School no. XIV}{Warsaw, PL}{}{}
\cvline{}{}
One of the best Polish high schools with strong mathematical tradition giving strong background for engineering studies.
The experimental mathematical program is introduced by best University teachers which gave strong background for further engineering studies.
\cvline{}{}

\section{Experience}

\vspace{4mm}
\cventry{2015 - ongoing}{Ultrahaptics}{\newline R\&D Engineer}{Bristol, UK}{}{}
My responsobility involve mostly C++ programming and system architecture. 
The company is a small start up so the nature of what we do causes hevy variation between skills used.
I am working in small team dedicated to inhouse tools development. We are Agile based TDD environment.
Most of our tools are written using boost libraries as well as QT framework. Some of them involve other technologies.
I have been also responsible for development of internal simulation library. In my team I am also doing most of software architecturing.
\cvline{}{}

\vspace{4mm}
\cventry{2013 - 2014}{PA Consulting}{\newline Analyst in Software Practice}{Melbourne, UK}{}{}
During my current role I have worked for a number of projects, the most notable of those:
\begin{itemize}
\item Boot loader design – for one of our customers we have created the design for the software of boot loader, the aim of the project was creating secure method of delivering firmware to device and creating the protocol for doing so.
\item Image processing project – the project goal was development C++ and C\# windows embedded software for QA of their product. I have contributed to GUI changes and some of Image processing methodologies.
\item WiFi firmware development – Development of a firmware (C for dedicated hardware) as well as looking at process side of our customer practice. I was responsible for starting the proper TDD as well as learning customer test system and spreading the knowledge along team.
\end{itemize}
Additionally I was involved in creation and casting the series of the workshop about OOD and design patterns for teammates in my Company.
\cvline{}{}

\vspace{4mm}
\cventry{2012 - 2013}{Samsung Electronics}{\newline Software engineer}{Warsaw, PL}{}{}
First stage of cooperation with Samsung company was strongly focused on training. During my first month of work I was assigned
to training task connected with linux kernel development, the project introduce creation of module tracing system calls usage from user space.
I have also took a part in the porting and driver development of linux, arm assembler and arm architecture courses.
Later I was reassigned for inner tool project - offline analyzer of dump states, the project was written in C++ using QT libraries
with parallel process comunication. After final stage I was responsible for designing tool for retrieving ramdumps from phones using
the USB communication. Currently I am not only developing inner tools but I have also became responsible for new model camera driver.
This role introduced new responsibilities - usage of p4v version manager software, usage and understanding of linux build system and
capability to fast bug tracing based on logs and offline debugging.
\cvline{}{}

\vspace{4mm}
\cventry{2011 - 2012}{Reynolds medical}{\newline Service engineer}{Warsaw, PL}{}{}
\cvline{}{}
The key responsibility during cooperation for Reynolds medical company was diagnosing and fixing hardware issues for medical devices.
The product galore strongly influenced my technical documentation and electronic schematics understanding. And the field on which
I have worked created need of medical and electronic device testing standard familiarity.
\cvline{}{}

\vspace{4mm}
\cventry{Summer 2010}{IPPT PAN}{\newline Intern}{Warsaw, PL}{}{}
\cvline{Project Developer}{Ram retrieving state machine communicating with windows driver}
\cvline{Project Developer}{PHP interface for ultrasound calibration machine}
\cvline{}{}
During the internships I was first assigned to PHP development of results presentation for ultrasound calibration machine.
After finishing my project my internships were elongated to 3 months, I had chance to took a part in development of Ultrasound
Doppler flowmeter. I was responsible for developing VHDL state machine for debugging RAM as DMA, and development of Windows HID driver
to communicate with my state machine.
\cvline{}{}

\vspace{4mm}
\cventry{Summer 2009}{GE}{\newline Java Programmer Intern}{Warsaw PL}{}{}
\cvline{}{}
During my internships for GE aviation in Polish Aviation Institute I was responsible for development of inner Java tool.
The program was analyzing advanced model grid data to help heat transfer team retrive lost data after analizer tool had crushed.
\cvline{}{}

\vspace{4mm}
\cventry{Summer 2008}{Biside}{}{\newline Warsaw, PL}{}{}
\cvline{}{}
During first period of cooperation with Biside company I was responsible for PHP front side development for telecommunication databases.
Later I was reassigned to SQL build manager system, my key role was creation, development and running test scenarios for the project.
\cvline{}{}

\section{Other projects}
\vspace{4mm}
\cventry{Spring 2013}{Linde gas}{}{}{}{}
\cvline{}{}
Concept design, creation and development of stocktaking management web tool in Django python framework.
\cvline{}{}

\section{Activities}
\vspace{4mm}
\cventry{}{Current Hobbies}{}{}{}{ Guitar, Violin, Rock Climbing, Miniature painting, Board games}

\end{document}
