%% start of file `template.tex'.
%% Copyright 2006-2010 Xavier Danaux (xdanaux@gmail.com).
%% Copyright 2010-2011 Mark Liu (markwayneliu@gmail.com).
%
% This work may be distributed and/or modified under the
% conditions of the LaTeX Project Public License version 1.3c,
% available at http://www.latex-project.org/lppl/.

\documentclass[8pt,a4paper]{moderncv}

\usepackage{verbatim}

% moderncv themes
\moderncvtheme[blue]{classic}                 % optional argument are 'blue' (default), 'orange', 'red', 'green', 'grey' and 'roman' (for roman fonts, instead of sans serif fonts)
%\moderncvtheme[green]{classic}                % idem

% character encoding
\usepackage[utf8]{inputenc}                   % replace by the encoding you are using

% adjust the page margins
\usepackage[scale=0.8]{geometry}
%\setlength{\hintscolumnwidth}{3cm}						% if you want to change the width of the column with the dates
%\AtBeginDocument{\setlength{\maketitlenamewidth}{6cm}}  % only for the classic theme, if you want to change the width of your name placeholder (to leave more space for your address details
%\AtBeginDocument{\recomputelengths}                     % required when changes are made to page layout lengths


% personal data
\firstname{Mateusz}
\familyname{Starzycki}
\address{36c Untere Parkstrass}{85540 Haar}    % optional, remove the line if not wanted
\mobile{+48 601 893 864}                    % optional, remove the line if not wanted
\email{mstarzycki@gmail.com}                      % optional, remove the line if not wanted
%\extrainfo{\url{http://markliu.me}} % optional, remove the line if not wanted

% to show numerical labels in the bibliography; only useful if you make citations in your resume
%\makeatletter
%\renewcommand*{\bibliographyitemlabel}{\@biblabel{\arabic{enumiv}}}
%\makeatother

%\nopagenumbers{}                             % uncomment to suppress automatic page numbering for CVs longer than one page
%----------------------------------------------------------------------------------
%            content
%----------------------------------------------------------------------------------

\begin{document}
\maketitle

\section{Skills}
\subsection{Proficient With}
\cvline{Build Tools}{Power user of Git, Docker and claud based build systems. Familiar with DevOps principles.}
\cvline{C}{Kernel source reading. Driver maintenance and development. Low level design. Bare metal proramming. Object oriented low level C programming principles.}
\cvline{C++}{Development using boost and QT libraries. Stl, contributor to Qt linguist. Familiar with Protobuf, modern design techniques and design patterns. User of conan, CMake, Make. Developed multiple reusable libraries for various projects.}
\cvline{Javascript}{Usage in QML in Qt. Development of front and backend node website. Familiar with redux, react and modern javascript.}
\cvline{Python}{Commercial Django project. University project under PyGTK for raspberry Pi board.}
\cvline{Technologies}{Driver development, Django, node, mongoDB, javascript, PostgreSQL, MySQL, \LaTeX{}, Bash Scripting, Git, Vim, Ubuntu, Fedora, POSIX, Gcc, Perforce, stdlib, Linux libraries}
\subsection{Have Experience With}
\cvline{Languages}{PHP, Java, VHDL}
\cvline{Technologies}{Apache, SQL Server, Eclipse, Raspdebian, Windows, Boost, Samba, Make, CMake, GDB, ADB, ARM Assembler}

\section{Languages}
\cvline{Polish}{Native}
\cvline{English}{Good speaking an writing skill}
\cvline{German}{Medium speaking and writing skill}

\section{Education}
\vspace{4mm}
\cventry{2008--2013}{Bachelor of Engineering, Mechatronics Studies}{\newline Warsaw University of technology}{Warsaw, PL}{}{}
\cvline{Advisor:}{\small Professor Jakub Zmigrodzki}
\cvline{Thesis:}{Graphical user interface for experimental Doppler Flowmeter based on ARM micro controller}
\cvline{}{}
\vspace{4mm}
\cventry{2005--2008}{Experimental Mathematical Program class}{\newline Stanisław Staszic High School no. XIV}{Warsaw, PL}{}{}
\cvline{}{}
\cvline{}{}

\section{Experience}

\vspace{4mm}
\cventry{2019-present}{eGym}{\newline (6 month contract) Software Contractor}{Munich, GE}{}{}
At eGym I am responsible for delivering a embedded QML application for their new product which is about to hit market later this year.
I am responsilbe for everyday feature development in the QT, crating the software archtiecture, maintaining and developing communication libraries.
I have also been involved in refining their software process regarding the reusable libraries and their maintenance.
\cvline{}{}

\vspace{4mm}
\cventry{2019}{Semcon}{\newline (3 month contract) Software Contractor}{Kongsberg, NO}{}{}
I was responsible for creating the architecture and developing QT based application for one of the Oil and Gas customers.
The application was monitoring and changing the settings of a customer embedded hardware.
The multiple modes of inter-device communication were involved.
I was mainly responsible for delivering the reusable QML components for the program, creating the architecture and domain libraries.
\cvline{}{}

\vspace{4mm}
\cventry{2019}{Mapscape}{\newline (6 month contract) Software Contractor}{Eindhoven, NL}{}{}
During mapscape contract I was responsible for maintenance and development of new features in map compilation software. I was using the C++ library as an interface to SQL lite to define refinement processes for creating the new map features.
\cvline{}{}

\vspace{4mm}
\cventry{2017 - 2018}{TomTom}{\newline (18 months Contract) Senior Software Engineer}{Amsterdam, NL}{}{}
2017-Present TomTom,
Senior Software Engineer, Amsterdam, Netherlands.
As a Senior software engineer in the navigation rendering team I was responsible for helping with feature develop-
ment and bug fixing in the team. I helped design and develop multiple changes in the rendering pipeline. I also helped to sketch out the plan for improvement of CI on the Android platform, created custom service for better test
system monitoring.
\cvline{}{}

\vspace{4mm}
\cventry{2017}{Samsung}{\newline (6 months contract) Software Engineer}{Warsaw, UK}{}{}
My responsibility involved mostly C++ programming and system maintenance. I was responsible for helping out
the in-house team in everyday job. I was placed in a team responsible for the Samsung Smart TV middleware. Everyday job composed of bugfixes and feature request development for Linux based multi-platform firmware.
During my time in Samsung I helped to refactor the sqlite management module.
\cvline{}{}

\vspace{4mm}
\cventry{2015 - 2016}{Ultrahaptics}{\newline (22 months) R\&D Engineer}{Bristol, UK}{}{}
My responsibility involved mostly C++ programming and system architecture. The company is a small startup
so the nature of what we do causes heavy variation between skills used. I was working in small team
dedicated to in-house tools development. We are Agile based TDD environment. Most of our tools are
written using boost libraries as well as QT framework. Some of them involve other technologies. I have been
also responsible for development of internal simulation library. In my team I was also doing most of
software architecting.
\cvline{}{}

\vspace{4mm}
\cventry{2013 - 2014}{PA Consulting}{\newline Analyst in Software Practice}{Melbourne, UK}{}{}
During my current role I have worked for a number of projects, the most notable of those:
\begin{itemize}
\item Boot loader design for one of our customers we have created the design for the software of boot loader, the aim of the project was creating secure method of delivering firmware to device and creating the protocol for doing so.
\item Image processing project the project goal was development C++ and C\#windows embedded software for QA of their product. I have contributed to GUI changes and some of Image processing methodologies.
\item WiFi firmware development. Development of a firmware (C for dedicated hardware) as well as looking at process side of our customer practice. I was responsible for starting the proper TDD as well as learning customer test system and spreading the knowledge along team.
\end{itemize}
Additionally, I was involved in creation and casting the series of the workshop about OOD and design patterns for
teammates in my Company.
\cvline{}{}

\vspace{4mm}
\cventry{2012 - 2013}{Samsung Electronics}{\newline (13 months) Software engineer}{Warsaw, PL}{}{}
First stage of cooperation with Samsung company was strongly focused on training. During my first month of
work I was assigned to training task connected with Linux kernel development, the project introduce creation of
module tracing system calls usage from user space. I have also taken part in the porting and driver development
of Linux, arm assembler and arm architecture courses. Later I was reassigned for inner tool project - offline
analyzer of dump states, the project was written in C++ using QT libraries with parallel process communication.
After final stage I was responsible for designing tool for retrieving ramdumps from phones using the USB
communication. Currently I am not only developing inner tools but I have also became responsible for new model
camera driver. This role introduced new responsibilities - usage of p4v version manager software, usage and
understanding of Linux build system and capability to fast bug tracing based on logs and offline debugging.
\cvline{}{}

\vspace{4mm}
\cventry{2008- 2012}{Various interns and other jobs}{\newline (~ 2.5 year)}{Warsaw, PL}{}{}
Php development, SQL data base development, Software development in Java for embedded system.
VHDL design, service maintenance - physical device assembly and debugging skills.

\cvline{}{}

\section{Activities}
\vspace{4mm}
\cventry{}{Hobbies}{}{}{}{ Violin, Rock Climbing, Miniature painting, Board games, Sailing, Math and Physics enthusiast}

\end{document}
